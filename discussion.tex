\section{Discussion -- addressing perceived limitations}
\label{sec:disc-perc-limit}

We dedicate this section to discuss the details of the delay semantics,
the dependence on time analyzable platforms, and the various uses and
programming paradigms that can be built using the \texttt{delay (M..N)}
construct.

\subsection{Programming using the delay construct}
\label{sec:progr-using-delay}

All three delay constructs are useful for programming real systems. % The
% first form of the delay construct $delay(M..N)$ is used to program
% non-determinism like in the HRCTS system and a real printer-spooler
% embedded system, etc. The second form is useful for programming low
% priority periodic tasks.

\subsubsection{Non-deterministic time}
\label{sec:non-determ-time}

Many real-world systems require non-deterministic timing
constructs. Where the exact real-time delay is not known or should not
be known apriori. One such system is our motivating example -- the human
response time system (HRTCS). Another one we presented was a real
printer-spooler embedded controller in
Section~\ref{sec:experimental-results}.

\subsubsection{Timeout}
\label{sec:timeout}

There are many instances when one would like to wait on an input from
environment for only a specified amount of time. This can be programmed
as:

\begin{figure}[h!]
  \centering
    \begin{minipage}[b]{\linewidth}
\begin{verbatim}
// timeout after 1ms.
trap(T){{await(A);}||
        {delay(1 ms);exit(T);}}
\end{verbatim}
    \end{minipage}
  \caption{Programming low priority timeout tasks with delay statements}
  \label{fig:timeout}
\end{figure}

In the above case, the program waits for signal \texttt{A} from the
environment for \textit{at least} \texttt{1 ms}, but it may wait longer
if other reactions/CDs are scheduled for execution. This second variant
of delay statement is very useful for programming low priority tasks.

\subsubsection{Periodic reactions}
\label{sec:periodic-reactions}

The third variant is useful for hard-real time guarantees, e.g., a
reaction, or a whole CD, can be programmed to run periodically like so:

\begin{figure}[h!]
  \centering
  \begin{minipage}[b]{\linewidth}
\begin{verbatim}
while(true) { delay ((1..1)ms); emit S; 
 //do something }
\end{verbatim}
  \end{minipage}
  \caption{Programming periodic tasks with delay statements}
  \label{fig:periodic}
\end{figure}

The above program will emit signal \texttt{S} every \texttt{1 ms}. A
periodic reaction (or CD) requires special consideration, like any other
real-time periodic task. One should be aware of the interaction of
delays and other \texttt{pause} constructs in the body of the reaction
(or CD). For example, consider the (perceived) periodic program and its
rewrite below.

\begin{figure}[h!]
  \centering
  \begin{SubFloat}{\label{pp:a}Original program}
    \centering
    \begin{minipage}[b]{0.3\linewidth}
\begin{verbatim}
while(true) {
 delay(1..1 ms);
 //extra pause
 pause;
 emit S;
}
\end{verbatim}
  \end{minipage}
\end{SubFloat}
\hspace{9pt}
  \begin{SubFloat}{\label{pp:b}Rewritten program}
    \begin{minipage}[h]{0.5\linewidth}
\begin{verbatim}
while(true) {
 while(true) { 
  trap(T) {
   int x = 0;
   while(true) {
    x = x + 1;
    pause;
    if (x == d)
     exit(T);
   }
  }
  //extra pause
  pause;
  emit S;
 }
}
\end{verbatim}
    \end{minipage}
  \end{SubFloat}
\caption{Extra pause statements in periodic tasks}
  \label{fig:periodic2}
\end{figure}

The system designer needs to be aware that just introducing a delay
construct does not make the reaction (or CD) periodic. The program on
the left does \textit{not} emit signal \texttt{S} every \texttt{1 ms},
because of the extra \texttt{pause} statement. Introducing additional
pauses introduces more logical ticks, consequently increasing the
real-time delay. Static analysis techniques from real-time community can
be used to automatically determine the delay specification in such
cases, but describing these techniques is out of the scope of this work.

\subsection{Need for a time analyzable platform -- interaction of
  preemption and delays}
\label{sec:inter-preempt-delays}

As mentioned previously, the proposed solution might need
over-approximation techniques (Algorithm~\ref{alg:2}) to guarantee
timing requirements (cf. Section~\ref{sec:extend-tehcn-vari}). This
might give the readers an impression that timer based solution proposed
for standard real-time operating systems should be utilized for
GALS/synchronous programs. This is a fallacy. Bourke et
al~\cite{Bourke2009a} have investigated this technique
unsuccessfully. We elaborate upon the interplay of timers and reactivity
to make the problem of using timers explicit.

Preemption plays an important role in reactive languages. One needs to
carefully consider the interplay of \texttt{delay} semantics with the
preemption semantics of reactive languages. Previous attempts at
incorporating delays (using external timers) have only been partially
successful, because of the complex interplay between real-time and
preemption. Consider the simple example below, which models real-time
using external timers as in~\cite{rsh94}. 

\begin{figure}[h!]
  \centering
  \begin{minipage}[b]{\linewidth}
\begin{verbatim}
suspend(S) {
 emit START_TIMER(10); 
 await (TIMER);emit O1;
}
\end{verbatim}
  \end{minipage}
  \caption{Interaction of preemption and external timers}
  \label{fig:preemp}
\end{figure}

As identified in~\cite{Bourke2009a} \texttt{suspend} does not play well
with the external timer. The above program sends a signal to an external
timer and waits for 10 ms to pass by. Consider what happens when signals
\texttt{S} and \texttt{TIMER} occur in the same logical tick, the
\texttt{await} statement is never executed (due to suspend) and hence,
we enter a deadlock. Such problems are completely avoided in our
technique, because we convert the real-time delays into logical delays
(\texttt{pause} constructs), which interact well with preemption.
Furthermore, synchronous/GALS languages have always been targeted at
real-time analyzable platforms and constrained
environments~\cite{DBLP:journals/pieee/SifakisTY03,boldt07}, one can use
our technique within such a setting with ease.

% \subsection{Other semantic discussion}
% \label{sec:other-semant-dissc}

\subsection{Interaction of channel communication and delays}
\label{sec:inter-chann-comm}

Channels, used for communication between reactions in asynchronously
running CDs, are an addition in the SystemJ language. Like interaction
of preemption and delays, conversion of real-time delays to logical
ticks also interacts well with channel rendezvous, because the semantics
of interaction are well defined~\cite{amal10}. More importantly, we need
to consider the interplay of channel communication and WCRT/BCRT
analysis. Since, channel communication does not stop logical time
(see~\cite{amal10}) WCRT/BCRT are unaffected by channel
communication. % The response time to input signals \textit{is} though!
% But, we are unconcerned with the response time, analysis in this paper
% and it remains a future research avenue.

% \subsubsection{Interaction of synchronous parallelism and delay}
% \label{sec:inter-synchr-parall}

% {\color{red}Maybe we should talk about how the control flow in other
%   reactions and CDs is not affected by the delay statement.}



%%% Local Variables: 
%%% mode: latex
%%% TeX-master: "paper"
%%% End: 
