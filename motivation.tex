\subsection{Motivating Example}
\label{sec:motivating-example}

% system{
%  interface{
%   input signal TOUCH;
%   output signal GREEN_LIGHT;
%   output char channel S;
%   input char channel R;
%  }
\begin{figure}[t!]
% \begin{center}
	\vspace{-10pt}
		\begin{lstlisting}[style=sysj,morekeywords={sustain,send,receive,abort,await,emit,present,trap,pause,exit,delay,suspend}]
{ // Clock-domain 1
while(true) {
  trap(TRAP){
   // Abort if user touches the screen
   abort(TOUCH){
    {sustain GREEN_LIGHT;} //reaction R1
    || // synchronous parallel operator
    {
     //exit after any where from 50.3 to 200.3 ms.
     delay (50.3 .. 200.3 ms); 
     exit(TRAP); 
    } // reaction R2
   } do {send S(1);} // Send 1 if touched 
  } do {send S(0);} // Else Send 0
  pause; // only construct to consume logical time
 }
}
>< // asynchronous operator
{ // Clock-domain 2
  while(true){ receive S; // do something with S}
}
\end{lstlisting}
\caption{An example human responsiveness system (HRTCS)}
% }
% \end{center}
\label{fig:1}
\end{figure}

Consider a machine used to test human response time and collect this
data for future research purposes. The machine switches on a green light
on a touch screen for anywhere between 50.3 ms to 200.3 ms, if the user
can touch this green light then she is successful and an integer
\texttt{1} is sent to a database, if unsuccessful a \texttt{0} is sent
instead. Different statistics and analyses on human responsiveness can
be based on the collected data.

The SystemJ pseudo-code~\cite{amal10} for such a system is shown in
Figure~\ref{fig:1}. We use the SystemJ language in this paper, first of
all, because being a \textit{Globally Asynchronous Locally Synchronous}
(GALS) language it helps us easily describe a larger class of systems
(like HRTCS in Figure~\ref{fig:1}) compared to purely synchronous
languages like Esterel. Moreover, being a super-set of Esterel, SystemJ
allows us to explore the marriage of real-time and logical time, which
is equally applicable to the synchronous sub-set.

The SystemJ program is divided into two clock-domains, the first
clock-domain continuously displays a green light (\texttt{GREEN\_LIGHT}
signal) on the touch screen sensor and waits for the user to respond. If
the user is able to touch the screen within the specified time of 50.3
ms to 200.3 ms a positive response is sent to the second
clock-domain. All the SystemJ programming constructs used to implement
this system are described in~\cite{amal10} and Table~\ref{tab:1} in
Section~\ref{sec:background}. The delay specification, which models the
non-deterministic delay between 50.3 ms and 200.3 ms is not supported in
the current language and cannot be compiled with the current SystemJ
compiler. We introduce real-time \texttt{delay} statements in SystemJ.

%%% Local Variables: 
%%% mode: latex
%%% TeX-master: "paper"
%%% End: 
