\section{Related work}
\label{sec:related-work}


A small number of studies have been performed when it comes to
integrating real-time directly as a programming construct in GALS (and
its subset synchronous) reactive languages. The most prominent works in
this area are by Shyamsundar~\cite{rsh94} and Bourke et
al.~\cite{Bourke2009a}. Shyamsundar incorporates real-time using
external timers in \textit{Communicating Reactive Processes} (CRP),
which like SystemJ is an extension of synchronous language Esterel to
asynchorny. But, as mentioned in Section~\ref{sec:inter-preempt-delays},
external timers do not interact well with preemption constructs in these
languages. Bourke et al. do a much better job of introducing real-time
as first class constructs in the Esterel language. They like us provide
real-time \texttt{delay} as first class Esterel programming constructs,
and translate them into Esterel kernel constructs. But, unlike us they
do not translate delays into \texttt{pause} constructs
directly. Instead, logical ticks are generated by using abstract notion
of \texttt{event} and \texttt{sample} platform dependent timers. This
notion makes it a complex and inflexible solution since the number of
platform timers with certain resolutions need to be determined and
present on the system for the solution to be realizable. Moreover,
non-deterministic real-time delays and integration with non-maximal
parallelism is not studied at all by them. Other works such as
Quartz~\cite{glog02} also incorporate delay statements in Esterel using
\texttt{pause} constructs, but again, non-maximal parallelism,
non-deterministic delays are not studied. Moreover, Quartz is targeted
at studying timing properties using model checkers rather than
implementation, like us.


%%% Local Variables: 
%%% mode: latex
%%% TeX-master: "paper"
%%% End: 


