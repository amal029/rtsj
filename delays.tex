\section{Introducing real-time in SystemJ}
\label{sec:intr-real-time}

We introduce a single \textit{derived} (built from kernel constructs)
statement called \mbox{\texttt{delay (M..N)}} in the SystemJ language
for real-time control. The resolution of the delay statement is of
secondary concern and is dependent upon the execution platform. 


\subsection{Semantics of the delay statement}
\label{sec:semant-delay-stat}

\begin{figure}[t!]
  \centering
  \includegraphics[scale=0.6]{./sem}
  \caption{Pictorial representation of the semantics of the delay
    statement and its variants}
  \label{fig:5}
\end{figure}

Given a SystemJ program: \texttt{delay (M..N); p}, where $M \in
\mathbb{Q}^{>0}$ and $N \in \mathbb{Q}^{>0}$, statement $p$ is executed
after real-time delay of $\tau$ units, such that, $M \leq \tau \leq N$.

We also introduce two variants of the derived statement \texttt{delay}.
\begin{enumerate*}
\item Given a SystemJ program \texttt{delay (M); p}, where $M \in
  \mathbb{Q}^{>0}$, statement $p$ is executed after real-time delay of
  $\tau$ units, such that, $M \leq \tau < \infty$. It is important to
  note that the lower bound of the delay construct $M$ is \textit{not}
  an exact delay, but rather the control is allowed to proceed to the
  next statement anytime after the delay of time $M$.
\item Given a SystemJ program \texttt{delay (M..M); p}, where $M \in
  \mathbb{Q}^{>0}$, statement $p$ is executed after real-time delay of
  $\tau$ units, such that, $M \leq \tau \leq M$. In this variant, the
  delay is exact.
\end{enumerate*}

Figure~\ref{fig:5} pictorially represents the semantics of these delay
constructs. The construct \texttt{delay (M..N)} requires that upon
execution the control-flow is suspended for \textit{at least} $M$ units
and \textit{at most} $N$ units, depicted as bounded box in
Figure~\ref{fig:5}. There is no upper bound for the second delay variant
\texttt{delay(M)}, which requires that the control-flow suspends for
\textit{at least} $M$ units, but the following statement can be executed
after some countable delay, depicted with the unbounded box in
Figure~\ref{fig:5}. Finally, \texttt{delay (M..M)} requires that there
is an \textit{exact} delay of $M$ units, i.e., the control-flow suspends
execution for \textit{at least} and \textit{at most} $M$ units.

\subsection{Rewriting the \texttt{delay} statement}
\label{sec:rewr-delay-stat}

The introduced \texttt{delay} construct is not a kernel construct, but a
\textit{derived} construct built from the kernel constructs
(Table~\ref{tab:1}) in SystemJ. Figure~\ref{fig:3} gives the rewrite of
the \texttt{delay} construct to kernel statements.

\begin{figure}[tb]
    \begin{minipage}{\textwidth}
      \begin{scriptsize}
\begin{verbatim}
trap(T){
 int x = 0;
 while(true){
  x = x + 1;
  pause;
  if(x == d) exit (T); //wait for "d" ticks
 } // d is the number of logical ticks calculated
} // by Algorithm 1
\end{verbatim}
      \end{scriptsize}
    \end{minipage}
    \caption{The rewrite of \texttt{delay} construct}
    \label{fig:3}
\end{figure}

The fundamental observation is that real-time is converted into logical
time via the \texttt{pause} construct. The rewrite basically maps the
physical notion of time back to the elegant logical notion of time. The
rewrite \textit{delays} a certain number of logical ticks, before
proceeding to the next statement. The number of logical ticks \texttt{d}
to \textit{delay} (Figure~\ref{fig:3}) is determined by the compiler
statically at compile time. The value of \texttt{d} is intricately tied
with the WCRT and BCRT of the program and hence the execution platform.

\subsection{Finding the logical delay \texttt{d}}
\label{sec:find-logic-delay}

\begin{algorithm}[t!]
  \begin{minipage}{1.0\linewidth}
    \SetAlgoLined
    \KwData{WCRT, BCRT, $M \in \mathbb{Q}^{>0}$, $N \in \mathbb{Q}^{>0}$}
    \KwResult{d}
    let $l_1 \leftarrow \lceil \frac{M}{WCRT} \rceil$\;
    let $l_2 \leftarrow \lceil \frac{M}{BCRT} \rceil$\;
    let $u_1 \leftarrow \lfloor \frac{N}{WCRT} \rfloor$\;
    let $u_2 \leftarrow \lfloor \frac{N}{BCRT} \rfloor$\;
    let $F:(l_1,u_1) \rightarrow S_1$\;
    let $F:(l_2,u_2) \rightarrow S_2$\;
    let $D \leftarrow S_1 \cap S_2$\;
    \Return (some $d \in D$)\;
    \caption{Finding the value of \texttt{d}}
    \label{alg:1}
  \end{minipage}
\end{algorithm}

The computation of \texttt{d} is shown in Algorithm~\ref{alg:1}. This
algorithm is carried out for each CD (in SystemJ) or a synchronous
program individually. \textit{The fundamental observation is -- the
  reaction time for each logical tick is elastic -- varying only between
  the BCRT and the WCRT, thus any number of logical ticks \texttt{d}
  that map to the required real-time \texttt{delay} should be chosen in
  such a way that they are invariant to this elasticity.}

Our algorithm takes as input: WCRT, BCRT, and the lower and upper bounds
$M$ and $N$, respectively of the \texttt{delay} construct. We first
divide $M$ and $N$ with the BCRT and WCRT, respectively. This division
gives us the number of individual logical ticks required to delay the CD
(or synchronous program) by the real-time specification. We always
$ceil$ when dividing $M$ and $floor$ when dividing $N$ to make sure that
the resultant values are integers (in domain $\mathbb{N}^{>0}$) and
these functions guarantee that the resultant logical ticks result in
real-time delays between the required range $[M,N]$. Next, a function
$F$ maps these calculated values to a set of equidistant integer points
(values) separated by a unit value -- these points represent all the
logical ticks running at the WCRT and the BCRT, respectively that
satisfy the real-time delay requirements. The intersection of these two
sets gives all the logical ticks that satisfy the real-time requirements
invariant of the logical time and its elasticity.

Let us revisit our motivating example to illustrate the algorithm. From
Figure~\ref{fig:1} we know that $M$ is 50.3 ms and $N$ is 200.3 ms,
respectively. Let the WCRT and BCRT be: 0.112 ms and 0.0334 ms,
respectively. Thus, the algorithm proceeds as follows:

\begin{enumerate*}
\item $l_1 \leftarrow \lceil 50.3/0.112 \rceil$ and $u_1 \leftarrow
  \lfloor 200.3/0.112 \rfloor$. $l_1 = 450$ and $u_1 = 1788$. We first
  calculate the logical ticks that are always running at the WCRT and
  satisfy the required real-time delay.
\item $l_2 \leftarrow \lceil 50.3/0.0334 \rceil$ and $u_2 \leftarrow
  \lfloor 200.3/0.0334 \rfloor$. $l_1 = 1506$ and $u_1 = 5997$. We do
  the same for the BCRT case.
\item $S_1 = [450,1788]$ and $S_2 =[1506,5997]$. We then map the
  resultant bounds to linear points. Sets $S_1$ and $S_2$ represent
  logical ticks that running at the WCRT and BCRT, respectively always
  satisfy the required real-time constraints.
\item $D = S_1 \cap S_2$, $D = [1506,1788]$. Finally, the intersection
  of the two sets gives the set $D$ from which we can choose any value
  for \texttt{d}.
\end{enumerate*}

The resultant value for \texttt{d} gives the number of logical ticks,
which can run at any physical clock-speed, bounded by the BCRT and the
WCRT of the program and still result in the desired real-time delay. We
think this is an elegant solution, because the technique provides
real-time guarantees while preserving the essence (elastic logical tick)
of synchronous and GALS programming prescribed by SystemJ and Esterel
style languages. Moreover, this technique considers non
maximal-parallelism, i.e., the delay in logical ticks is calculated
after scheduling of CDs or synchronous programs has been performed. To
our knowledge we are the first to do so.

\subsection{Extending the technique to variants of \texttt{delay}}
\label{sec:extend-tehcn-vari}

\paragraph{The \texttt{delay(M)} construct}
\label{sec:extend-techn-vari}

The \texttt{delay(M)} variant is easily accommodated in the
technique. All one needs to do is find the set $S_1$ and choose a value
from this set.

\paragraph{The \texttt{delay(M..M)} construct}
\label{sec:extend-techn-vari}

The \texttt{delay(M..M)} variant is a little more interesting. Like
before we find sets $S_1$ and $S_2$, and find the intersection of the
two sets to get the value of \texttt{d}. It is possible (and often
likely, as suggested by our experiments in
Section~\ref{sec:experimental-results}) that the resultant set $D$ is
empty (also possible in the case of \texttt{delay(M..N)}, but never
possible in case of \texttt{delay(M)}). Note that $S_1$ and $S_2$ can
never be empty sets and will always contain at minimum a single
element. In such a case, we automatically \textit{relax} the upper bound
of the \texttt{delay} statement.

\paragraph{Relaxation of the upper real-time bound}
\label{sec:over-appr-relax}

The relaxation technique is shown in Algorithm~\ref{alg:2}.
  
\begin{algorithm}[t!]
  \begin{minipage}{1.0\linewidth}
    \SetAlgoLined
    \KwData{$S_2$, $D$, WCRT}
    \KwResult{d}
    \If {$D = \emptyset$} {
      let $j_{0}$ be the first element of set $S_2$\;
      \ShowLn let $N \leftarrow WCRT \times j_0$\;
      let $d \leftarrow j_0$\;
    }
    \Return d\;
    \caption{Calculating the minimum relaxation of the upper real-time
      bound}
    \label{alg:2}
  \end{minipage}
\end{algorithm}


This algorithm results in the \underline{smallest} relaxation required
for the real-time delay to be satisfied. The algorithm takes as input
the WCRT and set $S_2$, recall that set $S_2$ represents the logical
ticks required to satisfy the real-time requirement at the BCRT. We take
the very first value from set $S_2$ and multiply it with the WCRT to get
the relaxation $N$. The first element of set $S_2$ is returned as the
logical tick delay \texttt{d}. The fundamental observation is that we
have to delay for at least $M$ units of real-time, hence,
under-approximation is out of question. We can still over-approximate,
but to reduce the resultant error, we should over-approximate by least
possible value, which is the lower bound of set $S_2$. Thus, the lower
bound is considered to be the only element shared between the two sets
$S_1$ and $S_2$ and accordingly, the upper real-time bound is relaxed by
the multiplication of WCRT and the first element of set $S_2$.

For example, assume $BCRT=5$ and $WCRT=100$, with a required delay of
$(200..200)$. Algorithm~\ref{alg:1}, results in $D$ being an empty set,
since $S_1=[2]$ and $S_2=[40]$, respectively. In such a case, the
resultant relaxation is: $40 \times 100$ using Algorithm~\ref{alg:2},
which results in a time of $4000$ units. Such large over-approximations
can be avoided by breaking the critical paths (worst case ticks) in the
program (by inserting \texttt{pause} constructs), just like in hardware
design, which is very \textit{unsurprising} considering Esterel has its
origins in hardware.

\subsection{The tool-chain flow}
\label{sec:tool-chain-flow}

\begin{figure}[h!]
  \centering
  \includegraphics[scale=0.45]{./tool.pdf}
  \caption{The tool-chain flow of SystemJ compiler with delays}
  \label{fig:4}
\end{figure}

The interaction of the low-level execution platform and the described
high-level technique of introducing real-time delays in the SystemJ
language is shown in Figure~\ref{fig:4}. We use static-analysis
techniques to find out the WCRT/BCRT of the programs automatically after
rewrites and standard compiler optimizations. We require timing
analyzable platforms, in our case \textit{Java Optimized Processor}
(JOP)~\cite{jop:jnl:jsa2007}. Moreover, since algorithms~\ref{alg:1}
and~\ref{alg:2} are performed after scheduling for different constraints
such as faster execution time, or reduced power consumption, we,
inherently satisfy the criteria of exploiting non-maximal
parallelism. The low-level analysis guarantees a precise WCRT/BCRT for
the logical tick, which is then used for calculating $d$ in
Figure~\ref{fig:3}.

Our compiler has a switch that enables over-approximation as described
in Section~\ref{sec:extend-tehcn-vari} provided the delay requirements
are not met. If this compiler switch is not enabled, the compiler simply
outputs an error indicating the relaxation required. The system designer
can then either set the compiler switch or change the program by
shortening the critical path by introducing \texttt{pause}
constructs. We show how introducing \texttt{pause} constructs can help
reduce the WCRT in Section~\ref{sec:experimental-results}.

%%% Local Variables: 
%%% mode: latex
%%% TeX-master: "paper"
%%% End: 

