\section{Introduction and motivation}
\label{sec:intr-motiv}

Reactive languages~\cite{gber931,amal10} are a class of programming
languages used for designing and implementing reactive systems, which
continuously respond to input from their environment. These languages
have been successfully used in programming a plethora of systems such as
fly-by-wire in Airbus~\cite{eairbus}, security surveillance
systems~\cite{amal121}, etc.  Usually these systems also need to meet
real-time constraints enforced by the environment.  Yet, the languages
used to program these systems do not support describing real-time
constructs as first class language constructs.  For example, one cannot
describe a simple real-time delay (postponement) of an operation for 0.2
ms. These languages are based on formal semantics, essential for the
formal reasoning about and verification of correctness of functional
properties of the developed programs, but leave nonfunctional properties
such as timing behavior as an implementation
detail~\cite{boldt07}. Arguably, rightly so, because physical time
cannot be incorporated without information about the underlying
execution platform.  But, one can still reason in terms of logical
time. These languages support discrete logic clock rather than a
discrete physical (real-time) clock. The real-time period of the
discrete logic clock (also referred to as a logic tick) is not fixed and
it is determined by the responsiveness of the program to the input
signals. Unlike a discrete physical clock, which has fixed real-time
period, the period of the logic clock is elastic. Although the timing
model with logic clock has worked well for reactive synchronous
languages in designing discrete systems, there is a need to introduce
real-time since, often, implementation models are in the continuous or
discrete real-time domain~\cite{DBLP:journals/pieee/SifakisTY03}.
Moreover, adding timing capabilities as the first class language
constructs create additional opportunity for formally verifying
functional and non-functional requirements before system deployment.

\subsection{Motivating Example}
\label{sec:motivating-example}

% system{
%  interface{
%   input signal TOUCH;
%   output signal GREEN_LIGHT;
%   output char channel S;
%   input char channel R;
%  }
\begin{figure}[t!]
% \begin{center}
\begin{minipage}{5cm}
  \begin{scriptsize}
    
\begin{verbatim}
{ // Clock-domain 1
while(true) {
  trap(TRAP){
   // Abort if user touches the screen
   abort(TOUCH){
    {sustain GREEN_LIGHT;}
    || // synchronous parallel operator
    {
     //exit after any where from 50.3 to 200.3 ms.
     delay (50.3 .. 200.3 ms);
     exit(TRAP); 
    }
   } do {send S(1);} // Send 1 if touched 
  } do {send S(0);} // Else Send 0
  pause; // only construct to consume time
 }
}
>< // asynchronous operator
{ // Clock-domain 2
  while(true){ receive S; // do something with S}
}
\end{verbatim}
  \end{scriptsize}
\end{minipage}
\caption{An example human responsiveness system (HRTCS)}
% }
% \end{center}
\label{fig:1}
\end{figure}

Consider a simple machine used to test human response time and collect
this data for future research purposes. The machine switches on a green
light on a touch screen for anywhere between 50.3 ms to 200.3 ms, if the
user can touch this green light then she is successful and an integer
\texttt{1} is sent to a database, if unsuccessful a \texttt{0} is sent
instead, Different statistics and analyses on human responsiveness can
be based on collected data.

The SystemJ pseudo-code~\cite{amal10} for such a system is shown in
Figure~\ref{fig:1}. We use the SystemJ language in this paper, first of
all, because being a \textit{Globally Asynchronous Locally Synchronous}
(GALS) language it helps us easily describe a larger class of systems
(like HRTCS in Figure~\ref{fig:1}) compared to purely synchronous
languages like Esterel. Moreover, being a super-set of Esterel, SystemJ
allows us to explore the marriage of real-time and logical time, which
is equally applicable to the synchronous sub-set.

The SystemJ program is divided into two clock-domains, the first
clock-domain continuously displays a green light (\texttt{GREEN\_LIGHT}
signal) on the touch screen sensor and waits for the user to respond. If
the user is able to touch the screen within the specified time of 50.3
ms to 200.3 ms a positive response is sent to the second
clock-domain. All the SystemJ programming constructs used to implement
this system are described in~\cite{amal10} and Table~\ref{tab:1} in
Section~\ref{sec:background}. The delay specification, which models the
non-deterministic delay between 50.3 ms and 200.3 ms is not supported in
the current language and cannot be compiled with the current SystemJ
compiler. We introduce real-time \texttt{delay} statements in SystemJ.

%%% Local Variables: 
%%% mode: latex
%%% TeX-master: "paper"
%%% End: 


The major motivations for introducing delay mechanism and at the same
time contributions of the paper are: (1) delays allow real-time based
synchronization between concurrent behaviors, (2) delays model relays on
the use of relative instead of absolute time, i.e., a delay in selected
time units is counted from the currently executing SystemJ statement and
(3) delays allow certain behaviours to having real-time features, while
the other behaviours use only logical time and (4) real-time delay does
not affect the model of logical time in SystemJ program as long as the
amount of delay is within boundaries that can be determined statically
by program analysis. All these as well as semantics of the delay
construct are discussed and illustrated in the remaining part of the
paper. Our contributions can be further refined as follows:
\begin{enumerate*}
\item Real-time is modeled in the $\mathbb{Q}^{>0}$ domain of numbers,
  i.e., non-negative rational numbers.
\item We consider the property of \textit{non
    maximal-parallelism}. \textit{Non-maximal parallelism:} There are
  not always sufficient resources for processes to execute, so
  effectively processes are allocated and scheduled with different
  optimization criteria -- reducing computation time, reduced power
  consumption, etc. This allocation and scheduling is an integral part
  of any language compilation and needs to be considered when
  introducing real-time delays.
% \item Our solution does not change the functional or timing semantics of
%   GALS and its subset synchronous programs, in fact, our solution does
%   not require one to change the mid-end, the back-end or the related
%   optimization phases of the compiler at all.
\item We are the first, to our knowledge, to allow specification of both
  deterministic and non-deterministic real-time in GALS and synchronous
  languages. \textit{Deterministic delay:} is the ability to specify a
  single \textit{real-time} delay. \textit{Non-deterministic delay:} is
  the ability to specify a range, of real-time values for a delay
  statement, e.g., Figure~\ref{fig:1}.
\end{enumerate*}

The rest of the paper is arranged as follows:
Section~\ref{sec:background} gives the background on SystemJ and related
techniques required for the rest of the
paper. Section~\ref{sec:intr-real-time} introduces real-time delays in
the GALS paradigm and describes their compilation into logical
time. Section~\ref{sec:experimental-results} gives the experimental
results for a set of real-time
applications. Section~\ref{sec:related-work} positions this work in
relation to other approaches to introducing real-time. Finally, we
conclude in Section~\ref{sec:concl-future-work}.


%%% Local Variables: 
%%% mode: latex
%%% TeX-master: "paper"
%%% End: 
