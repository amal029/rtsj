\subsection{Motivation via examples of real-time systems programmed in a
reactive language SystemJ}
\label{sec:motivating-example}

% system{
%  interface{
%   input signal TOUCH;
%   output signal GREEN_LIGHT;
%   output char channel S;
%   input char channel R;
%  }

Real-time operating systems~\cite{barry2009using} usually provide two
types of mechanisms for introducing real-time in the developed program:
(1) the ability to perform a timeout and (2) the ability to run tasks
periodically with some real-time period $T$. We would like to provide
similar mechanisms in reactive languages for real-time program
development. In fact, we provide \textit{exact} and \textit{non exact}
real-time control mechanisms -- a generalization of the two real-time
mechanisms introduced in real-time operating systems.

\subsubsection{Programming using the non exact real-time control
  construct; the \textrm{\texttt{wait\_inbetween}} statement}
\label{sec:progr-using-non}

\begin{figure}[t!]
	\vspace{-10pt}
        \begin{SubFloat}{\label{delaycode:a}A human responsiveness
            system (HRTCS)}
        \begin{lstlisting}[style=sysj,morekeywords={sustain,send,receive,abort,await,emit,present,trap,pause,exit,wait_inbetween,wait_exact,suspend}]
{ // Clock-domain 1
while(true) {
  trap(T){
   // Abort if user touches the screen
   abort(TOUCH){
    {sustain GREEN_LIGHT;} //reaction R1
    || // synchronous parallel operator
    {
     //exit after any where from 50.3 to 200.3 ms.
     wait_inbetween (50.3 .. 200.3 ms); 
     exit(T); 
    } // reaction R2
   } do {send S(1);} // Send 1 if touched 
  } do {send S(0);} // Else Send 0
  wait_exact(1 s); // Wait for 1 second
 }
}
>< // asynchronous operator
{ // Clock-domain 2
  while(true){ receive S; 
      if(#S == 1) counter++; else counter--;}
}
\end{lstlisting}
\end{SubFloat}
\begin{SubFloat}{\label{delaycode:b}Timing diagram -- the X-axis shows
    the execution of the \texttt{wait\_inbetween} statement. The Y-axis
    is the time elapsed.}
  % \begin{figure}[h!]
    \centering
    \includegraphics[scale=0.7]{FF1}
  % \end{figure}
\end{SubFloat}
\caption{Programming a human responsiveness system (HRTCS) with
  \texttt{wait\_inbetween}}
% }
% \end{center}
\label{fig:1}
\end{figure}

Consider a machine used to test human response time and collect this
data for future research purposes. The machine switches on a green light
on a touch screen for anywhere \textit{in between} 50.3 ms to 200.3 ms,
if the user can touch this green light then she is successful and \hj{an 
integer \texttt{1} is sent to other process in order to increase an
integer counter, if unsuccessful a \texttt{0} is sent instead to
decrease the counter}. Different statistics and analyses on human
responsiveness can be based on the collected data.

The SystemJ pseudo-code~\cite{amal10} for such a system, along with the
expected timing behavior of the \texttt{wait\_inbetween} statement, is
shown in Figure~\ref{fig:1}. We use the SystemJ language in this paper, 
because being a \textit{Globally Asynchronous Locally
  Synchronous} (GALS) language it helps us easily describe a larger
class of systems (like HRTCS in Figure~\ref{fig:1}) compared to purely
synchronous languages like Esterel. Moreover, being a super-set of
Esterel, SystemJ allows us to explore the marriage of real-time and
logical time, which is equally applicable to the synchronous sub-set.

The SystemJ program is divided into two \hj{mutually asynchronous}
clock-domains, the first clock-domain continuously displays a green
light (\texttt{GREEN\_LIGHT} signal) on the touch screen sensor and
waits for the user to respond. If the user is able to touch the screen
within the specified time of 50.3 ms to 200.3 ms a positive response is
sent to the second clock-domain. All the SystemJ programming constructs
used to implement this system are described in Table~\ref{tab:1} \hj{and
more details can be found in~\cite{amal10}}. The
\texttt{wait\_inbetween} specification, which models the non exact
delay between 50.3 ms and 200.3 ms is not supported in the current
language and cannot be compiled with the current SystemJ compiler. We
introduce such \texttt{wait\_inbetween} statements in SystemJ. Such
non exact real-time mechanisms are useful in developing systems where
the real-time delay is not known or should not be known a priori.

\subsubsection{Programming a timeout using the non exact real-time
  construct; the \texttt{wait\_atleast} statement}
\label{sec:progr-time-using}

\begin{figure}[b!]
	\centering
	\vspace{-10pt}
        \begin{SubFloat}{\label{dd:a}Timeout waiting for the input
            signal \texttt{DoorOpened}}
        \begin{lstlisting}[style=sysj,morekeywords={abort,await,emit,present,trap,pause,exit,wait_atleast,suspend}]
trap(T){
  abort(DoorOpened){
    wait_atleast(10000 ms);
    exit(T);  
  }
}
\end{lstlisting}
\end{SubFloat}
\begin{SubFloat}{\label{dd:b}Timing diagram -- the X-axis shows the
    execution of the \texttt{wait\_atleast} statement. The Y-axis is the
    time elapsed.}
\includegraphics[scale=0.7]{FF2}
\end{SubFloat}
\caption{Programming a timeout with \texttt{wait\_atleast}}
\label{dd}
\end{figure}

There are many instances when one would like to wait on an input from
environment for only a specified amount of time. If the signal is not
received then a timeout is generated so that the program can make
progress. Such a timeout programmed in the SystemJ pseudo-code along
with its timing behavior is shown in Figure~\ref{dd}. The SystemJ
program is waiting on an input signal, \texttt{DoorOpened}, from the
environment for 10 seconds and makes progress irrespective of the
reception of this signal after 10 seconds.

The timing diagram, Figure~\ref{dd:b}, shows that the
\texttt{wait\_atleast} statement waits for a \textit{at least} 10
seconds, but unlike \texttt{wait\_inbetween} there is no upper
bound. The statement following the \texttt{wait\_atleast} statement is
executed after a finite countable, but unbounded delay. Such a statement
is useful for programming low priority tasks.


\subsubsection{Programming periodic tasks using the exact real-time
  construct; the \texttt{wait\_exact} statement}
\label{sec:progr-using-exact}

The previous wait statements are non exact, i.e., the \hj{time of}
execution of the statement following the wait statement cannot be
controlled exactly. The final variant we introduce allows doing exactly
that. We call it \texttt{wait\_exact} and this statement can be used to
program exact timeouts, periodic tasks, etc.

\begin{figure}[t!]
  \centering
	\vspace{-10pt}
        \begin{SubFloat}{\label{pp:a}Periodically emitting signal S}
        \begin{lstlisting}[style=sysj,morekeywords={emit,trap,pause,exit,wait_exact}]
while(true) { 
 wait_exact (1 ms); 
 emit S; 
 //do something 
}
\end{lstlisting}
\end{SubFloat}

\begin{SubFloat}{\label{p1:b}Timing diagram -- the X-axis shows the
    execution of the \texttt{wait\_exact} statement. The Y-axis is the
    time elapsed.}
  \includegraphics[scale=0.7]{FF3}
\end{SubFloat}
  \caption{Programming a periodic task with wait\_exact}
  \label{fig:p1}
  \vspace{-10pt}
\end{figure}

One such example; periodic emission of a signal to the environment is
shown in Figure~\ref{fig:p1}. As before upon execution of
\texttt{wait\_exact (1 ms)} statement, the time starts elapsing. But,
unlike in previous cases the following statement needs to be executed
exactly after 1 ms have passed, as shown in Figure~\ref{p1:b}.

%%% Local Variables: 
%%% mode: latex
%%% TeX-master: "paper"
%%% End: 
