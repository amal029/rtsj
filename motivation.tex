\subsection{Motivating Example}
\label{sec:motivating-example}

% system{
%  interface{
%   input signal TOUCH;
%   output signal GREEN_LIGHT;
%   output char channel S;
%   input char channel R;
%  }
\begin{figure}[t!]
% \begin{center}
\begin{minipage}{5cm}
\begin{verbatim}
{ // Clock-domain 1
while(1) {
  trap(TRAP){
   // Abort if user touches the screen
   abort(TOUCH){
    {sustain GREEN_LIGHT;}
    || // synchronous parallel operator
    {
    //exit after any 
    // where from 0.5 to 1 ms.
    /*delay [0.5..1]ms;*/
    exit(TRAP); 
    }
   } do {send S(1);} // Send 1 if touched 
  } do {send S(0);} // Else Send 0
  pause; // only construct to consume time
 }
}
>< // asynchronous operator
{ // Clock-domain 2
  while(1){ receive S;}
  // do something with S
}
\end{verbatim}
\end{minipage}
\caption{An example human reaction time collection system (HRTCS)}
% }
% \end{center}
\label{fig:1}
\end{figure}

Consider a simple machine used to test human reaction time and collect
this data for future research purposes. The machine switches on a green
light on a touch screen for anywhere between \texttt{0.5ms} to
\texttt{1ms}, if the user can touch this green light then she is
successful and an integer \texttt{1} is sent to a data-base server, if
unsuccessful a \texttt{0} is sent instead, this way a database of human
responsiveness can easily be generated.

The SystemJ~\cite{amal10} code for such a system is shown in
Figure~\ref{fig:1}. We use the SystemJ language in this paper first of
all, because being a \textit{Globally Asynchronous Locally Synchronous}
(GALS) language it helps us easily describe a larger class of systems
(like HRTCS in Figure~\ref{fig:1}) compared to purely synchronous
languages like Esterel. Moreover, being a super-set of Esterel, SystemJ
allows us to explore the marriage of real-time and logical time, which
is equally applicable to the synchronous sub-set.

The SystemJ program is divided into two clock-domains, the first
clock-domain continuously displays a green light (\texttt{GREEN\_LIGHT}
signal) on the touch screen sensor and waits for the user to respond. If
the user is able to touch the screen within the specified time of
\texttt{0.5ms} to \texttt{1ms} a positive response is sent to the second
clock-domain, which is interacting with data-base (via CSP style channel
communication via \texttt{S}) server in Java. The synchronous parallel
operator \texttt{||} along with the \texttt{trap-exit} (exception
mechanism) is used to make sure that the green light is switched on for
only a specified delay. The 2 clock-domains are implemented using the
asynchronous parallel \texttt{><} operator. All these constructs are
described in~\cite{amal10}.

This SystemJ example is \textit{incomplete}, the \texttt{delay}
specification, which models the non-deterministic delay between
\texttt{0.5ms} and \texttt{1ms} does not work with the SystemJ compiler!
In fact, there is no way to include such real-time specifications in the
language itself.

%%% Local Variables: 
%%% mode: latex
%%% TeX-master: "paper"
%%% End: 
