\section{Proofs of correctness, completeness, and soundness}
\label{sec:proof-correctness}

Section~\ref{sec:intr-real-time} describes the techniques and algorithms
for introducing various types of wait statements providing real-time
control in the SystemJ language. In this section we give the formal
proof of correctness for the algorithms described in
Section~\ref{sec:intr-real-time}.

\begin{lemma}
  Algorithm~\ref{alg:1} gives a value for \texttt{d} such that: $\lceil
  \frac{M}{BCRT} \rceil \leq d \leq \lfloor \frac{N}{WCRT} \rfloor$
  provided $D$ is a non-empty set for construct \texttt{wait\_inbetween
    (M..N)}
\label{lemma:1}
\end{lemma}
\begin{proof}
  This lemma is trivially true from the definition of set $D$ in
  Algorithm~\ref{alg:1}.
\end{proof}

\begin{theorem}
  Given \texttt{wait\_inbetween (M..N)} Algorithm~\ref{alg:1} gives a
  value \texttt{d}, provided $D$ is a non empty set, such that for any
  given $BCRT \leq T \leq WCRT$, $M \leq T \times d \leq N$ holds.
\label{th:1}
\end{theorem}
\begin{proof}
  We use case analysis and contradiction to prove the above theorem.
  \begin{compactenum}[\hspace{0.25cm} 1.]
  \item Proof for case $T = WCRT$: Assume \mbox{$d \times WCRT < M$},
    then $d < \frac{M}{WCRT}$. $BCRT \leq WCRT$, by definition, hence,
    $\frac{M}{WCRT} \leq \frac{M}{BCRT}$. Thus for the assumption to
    hold, $d < \frac{M}{BCRT}$, which contradicts
    Lemma~\ref{lemma:1}. The assumption $d \times WCRT >N$ is trivially
    proven from Lemma~\ref{lemma:1}.
  \item Proof for case $T = BCRT$: Like above, the assumption $d \times
    BCRT < M$ is trivially proven from Lemma~\ref{lemma:1}. For $d
    \times BCRT > N$ to hold, $d > \frac{N}{BCRT}$ should hold. But,
    $BCRT \leq WCRT$, by definition, hence, $\frac{N}{BCRT} \geq
    \frac{N}{WCRT}$, which again contradicts Lemma~\ref{lemma:1}.
  \item Proof for case $BCRT < T < WCRT$: For $d \times T < M$ to hold,
    $d < \frac{M}{T}$ should hold. By definition, $T > BCRT$, thus,
    $\frac{M}{BCRT} < \frac{M}{T}$, hence, $d < {M}{BCRT}$ should hold,
    which contradicts Lemma~\ref{lemma:1}. We can prove the other case
    $d \times T \leq N$, using the upper bound from Lemma~\ref{lemma:1}
    similarly.
  \end{compactenum}
\end{proof}

Next, we prove the completeness property.
\begin{theorem}
  Given \texttt{wait\_inbetween (M..N)} Algorithm~\ref{alg:1} gives a
  value \texttt{d}, provided $D$ is a non empty set, such that for
  \textbf{all} given $BCRT \leq T \leq WCRT$, $M \leq T \times d \leq N$
  holds.
\end{theorem}
\begin{proof}
  From Theorem~\ref{th:1} case 3.
\end{proof}

Next, we sketch the proof for the opposite of completeness -- the
soundness property of Algorithm~\ref{alg:1}.

\begin{theorem}
  Given \texttt{wait\_inbetween (M..N)} Algorithm~\ref{alg:1}
  \textbf{does not} give a value \texttt{d}, such that for \textbf{all}
  given $BCRT > T > WCRT$, $M \leq T \times d \leq N$ holds.
\end{theorem}

\begin{proof}
  The proof is the opposite of case 3 in Theorem~\ref{th:1}.
\end{proof}

The proofs for correctness, completeness, and soundness for
Algorithm~\ref{alg:2} are similar to those of Algorithm~\ref{alg:1}.

%%% Local Variables: 
%%% mode: latex
%%% TeX-master: "paper"
%%% End: 
