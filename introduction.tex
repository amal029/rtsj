\section{Introduction and motivation}
\label{sec:intr-motiv}

Reactive languages~\cite{gber05,amal10} are a class of programming
languages used for designing and implementing \textit{reactive} systems,
which continuously respond to input from their environment. These
languages have been successfully used in programming a plethora of
systems such as fly by wire in Airbus~\cite{eairbus}, security
surveillance systems~\cite{amal121}, etc. Usually the working
environment of these reactive systems is also real-time. Yet, these
languages do not support describing real-time as first class language
constructs. For example, one cannot describe a simple real-time delay
\texttt{delay 0.2 ms} in these languages. These languages are based on
formal semantics, essential for the verification of functional
correctness properties of the developed programs, but leave
non-functional properties such as timing behavior as an implementation
detail~\cite{ringler00,boldt07}. Arguably, rightly so, because physical
time cannot be incorporated without information about the underlying
execution platform. But, one can still reason in terms of
\textit{logical} time. These languages support discrete \textit{logical}
clock rather than a continuous clock. The period of the discrete logical
clock (also referred to as a \textit{logical tick}) is determined by the
responsiveness of the program to the input signals and as opposed to a
discrete \textit{physical} clock, is elastic. Although this timing model
has worked well for these languages in designing discrete systems, there
is obviously a need to introduce continuous time since often
implementation models are in the continuous time
domain~\cite{DBLP:journals/pieee/SifakisTY03}.

\subsection{Motivating Example}
\label{sec:motivating-example}

% system{
%  interface{
%   input signal TOUCH;
%   output signal GREEN_LIGHT;
%   output char channel S;
%   input char channel R;
%  }
\begin{figure}[t!]
% \begin{center}
\begin{minipage}{5cm}
  \begin{scriptsize}
    
\begin{verbatim}
{ // Clock-domain 1
while(true) {
  trap(TRAP){
   // Abort if user touches the screen
   abort(TOUCH){
    {sustain GREEN_LIGHT;}
    || // synchronous parallel operator
    {
     //exit after any where from 50.3 to 200.3 ms.
     delay (50.3 .. 200.3 ms);
     exit(TRAP); 
    }
   } do {send S(1);} // Send 1 if touched 
  } do {send S(0);} // Else Send 0
  pause; // only construct to consume time
 }
}
>< // asynchronous operator
{ // Clock-domain 2
  while(true){ receive S; // do something with S}
}
\end{verbatim}
  \end{scriptsize}
\end{minipage}
\caption{An example human responsiveness system (HRTCS)}
% }
% \end{center}
\label{fig:1}
\end{figure}

Consider a simple machine used to test human response time and collect
this data for future research purposes. The machine switches on a green
light on a touch screen for anywhere between 50.3 ms to 200.3 ms, if the
user can touch this green light then she is successful and an integer
\texttt{1} is sent to a database, if unsuccessful a \texttt{0} is sent
instead, Different statistics and analyses on human responsiveness can
be based on collected data.

The SystemJ pseudo-code~\cite{amal10} for such a system is shown in
Figure~\ref{fig:1}. We use the SystemJ language in this paper, first of
all, because being a \textit{Globally Asynchronous Locally Synchronous}
(GALS) language it helps us easily describe a larger class of systems
(like HRTCS in Figure~\ref{fig:1}) compared to purely synchronous
languages like Esterel. Moreover, being a super-set of Esterel, SystemJ
allows us to explore the marriage of real-time and logical time, which
is equally applicable to the synchronous sub-set.

The SystemJ program is divided into two clock-domains, the first
clock-domain continuously displays a green light (\texttt{GREEN\_LIGHT}
signal) on the touch screen sensor and waits for the user to respond. If
the user is able to touch the screen within the specified time of 50.3
ms to 200.3 ms a positive response is sent to the second
clock-domain. All the SystemJ programming constructs used to implement
this system are described in~\cite{amal10} and Table~\ref{tab:1} in
Section~\ref{sec:background}. The delay specification, which models the
non-deterministic delay between 50.3 ms and 200.3 ms is not supported in
the current language and cannot be compiled with the current SystemJ
compiler. We introduce real-time \texttt{delay} statements in SystemJ.

%%% Local Variables: 
%%% mode: latex
%%% TeX-master: "paper"
%%% End: 


The fundamental contribution of this paper is the inclusion of real-time
constructs in GALS (and its subset synchronous) reactive
languages. Moreover, our contributions can be further refined as
follows:

\begin{enumerate*}
\item Real-time is modeled in the $\mathbb{Q}^{>0}$ domain of numbers,
  i.e., non-negative rational numbers.
\item We consider the property of \textit{non maximal-parallelism},
  i.e., we consider scheduling on constrained resources to be an
  integral part of the solution. To our knowledge we are the very first
  to do so.
\item Our solution does not change the functional or timing semantics of
  GALS and its subset synchronous programs, in fact, our solution does
  not require one to change the mid-end, the back-end or the related
  optimization phases of the compiler at all.
\item We are the first, to our knowledge, to allow specification of both
  deterministic and non-deterministic real-time in GALS and synchronous
  languages.
\end{enumerate*}

The rest of the paper is arranged as follows...


%%% Local Variables: 
%%% mode: latex
%%% TeX-master: "paper"
%%% End: 
