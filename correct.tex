\section{Proofs of correctness, completeness, and soundness}
\label{sec:proof-correctness}

Section~\ref{sec:intr-real-time} describes the techniques and algorithms
for introducing various types of wait statements providing real-time
control in the SystemJ language. In this section we give the formal
proof of correctness for the algorithms described in
Section~\ref{sec:intr-real-time}.

\begin{lemma}
  Algorithm~\ref{alg:1} gives a value for \texttt{d} such that: $\lceil
  \frac{M}{BCRT} \rceil \leq d \leq \lfloor \frac{N}{WCRT} \rfloor$
  provided $D$ is a non-empty set for construct \texttt{wait\_inbetween
    (M..N)}
\label{lemma:1}
\end{lemma}
\begin{proof}
  This lemma is trivially true from the definition of set $D$ in
  Algorithm~\ref{alg:1}.
\end{proof}

\begin{theorem}
  Given \texttt{wait\_inbetween (M..N)} Algorithm~\ref{alg:1} gives a
  value \texttt{d}, provided $D$ is a non empty set, such that for any
  given reaction time $\hj{\Delta}$ \hj{in the range of} $BCRT \leq
  \Delta \leq WCRT$, $M \leq \Delta \times d \leq N$ holds.
\label{th:1}
\end{theorem}
\begin{proof}
  We use contradiction to prove the theorem.
  \begin{compactenum}[\hspace{0.25cm} 1.]
  \item Proof for case $\Delta = WCRT$: Assume \mbox{$d \times WCRT <
      M$}, then $d < \frac{M}{WCRT}$. $BCRT \leq WCRT$, by definition,
    hence, $\frac{M}{WCRT} \leq \frac{M}{BCRT}$. Thus for the assumption
    to hold, $d < \frac{M}{BCRT}$, which contradicts
    Lemma~\ref{lemma:1}. The assumption $d \times WCRT >N$ is trivially
    proven from Lemma~\ref{lemma:1}.
  \item Proof for case $\Delta = BCRT$: Like above, the assumption $d
    \times BCRT < M$ is trivially proven from Lemma~\ref{lemma:1}. For
    $d \times BCRT > N$ to hold, $d > \frac{N}{BCRT}$ should hold. But,
    $BCRT \leq WCRT$, by definition, hence, $\frac{N}{BCRT} \geq
    \frac{N}{WCRT}$, which again contradicts Lemma~\ref{lemma:1}.
  \item Proof for case $BCRT < \Delta < WCRT$: For $d \times \Delta < M$
    to hold, $d < \frac{M}{\Delta}$ should hold. By definition, $\Delta
    > BCRT$, thus, $\frac{M}{\Delta} < \frac{M}{BCRT}$, hence, $d <
    \frac{M}{BCRT}$ should hold, which contradicts
    Lemma~\ref{lemma:1}. We can prove the other case $d \times \Delta >
    N$, using the upper bound from Lemma~\ref{lemma:1} similarly.
  \end{compactenum}
\end{proof}

Next, we prove the completeness property.
\begin{theorem}
  Given \texttt{wait\_inbetween (M..N)} Algorithm~\ref{alg:1} gives a
  value \texttt{d}, provided $D$ is a non empty set, such that for
  \textbf{all} given reaction times $\Delta$ in the range of $BCRT \leq
  \Delta \leq WCRT$, $M \leq \Delta \times d \leq N$ holds.
\end{theorem}
\begin{proof}
  Follows from Theorem~\ref{th:1}.
\end{proof}

Next, we sketch the proof for the soundness property of
Algorithm~\ref{alg:1}.

\begin{theorem}
  Given \texttt{wait\_inbetween (M..N)} Algorithm~\ref{alg:1}
  \textbf{does not} give a value \texttt{d}, such that for \textbf{all}
  given reaction times $\Delta$ in the range of $BCRT > \Delta > WCRT$,
  $M \leq \Delta \times d \leq N$ holds.
\end{theorem}

\begin{proof}
  Soundness is trivially proven knowing that there exists no $\Delta <
  BCRT$ or $\Delta > WCRT$ by definition
  (cf. Section~\ref{sec:mapping-logical-time}).
\end{proof}

The proofs for correctness, completeness, and soundness for
Algorithm~\ref{alg:2} are similar to those of
Algorithm~\ref{alg:1}. Furthermore, correctness, completeness, and
soundness for \texttt{wait\_atleast (M)} and \texttt{wait\_exact (M)}
are dependent on Theorem~\ref{th:1}, because these are variants of the
\texttt{wait\_inbetween (M..N)} statement.

%%% Local Variables: 
%%% mode: latex
%%% TeX-master: "paper"
%%% End: 
