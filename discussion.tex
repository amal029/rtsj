\section{Discussion}
\label{sec:disc-perc-limit}

We dedicate this section to discuss the details of the wait construct
semantics, their usage and the advantages and limitations of our
proposed technique to introducing these real-time constructs in a
reactive setting.

\subsection{Programming using the wait\_exactly construct -- the
  periodic reactions}
\label{sec:progr-using-delay}

% All three delay constructs are useful for programming real systems. % The
% first form of the delay construct $delay(M..N)$ is used to program
% non-determinism like in the HRCTS system and a real printer-spooler
% embedded system, etc. The second form is useful for programming low
% priority periodic tasks.

% \subsubsection{Non-deterministic time}
% \label{sec:non-determ-time}

% Many real-world systems require non-deterministic timing
% constructs. Where the exact real-time delay is not known or should not
% be known apriori. One such system is our motivating example -- the human
% response time system (HRTCS). Another one we presented was a real
% printer-spooler embedded controller in
% Section~\ref{sec:experimental-results}.

% \subsubsection{Timeout}
% \label{sec:timeout}

% There are many instances when one would like to wait on an input from
% environment for only a specified amount of time. This can be programmed
% as:

% \begin{figure}[h!]
%   \centering
% 			\vspace{-10pt}
% 			\begin{lstlisting}[style=sysj,basicstyle=\normalsize\ttfamily,morekeywords={await,emit,trap,pause,exit,delay}]
% // timeout after 1ms.
% trap(T){{await(A);}||
%         {delay(1 ms);exit(T);}}
% \end{lstlisting}
%   \caption{Programming low priority timeout tasks with delay statements}
%   \label{fig:timeout}
% 			\vspace{-10pt}
% \end{figure}

% In the above case, the program waits for signal \texttt{A} from the
% environment for \textit{at least} \texttt{1 ms}, but it may wait longer
% if other reactions/CDs are scheduled for execution. This second variant
% of delay statement is very useful for programming low priority tasks.

% \subsubsection{Periodic reactions}
% \label{sec:periodic-reactions}

% The third variant is useful for hard-real time guarantees, e.g., a
% reaction, or a whole CD, can be programmed to run periodically like so:

\begin{figure}[h!]
  \centering
	\vspace{-10pt}
        \begin{lstlisting}[style=sysj,basicstyle=\normalsize\ttfamily,morekeywords={emit,trap,pause,exit,wait_exactly}]
          while(true){wait_exactly (1 ms); emit S; //do something
          }
 \end{lstlisting}
  \caption{Programming periodic tasks with wait\_exactly statement}
  \label{fig:periodic}
	\vspace{-10pt}
\end{figure}

The program in Figure~\ref{fig:preemp}, reproduced from
Section~\ref{sec:progr-using-exact}, is supposed to emit signal
\texttt{S} every \texttt{1 ms}. Such periodic reactions (or CDs) require
special consideration, like any other real-time periodic task. One
should be aware of the interaction of wait and other \texttt{pause}
constructs in the body of the reaction (or CD). For example, consider
the (perceived) periodic program and its rewrite below.

\begin{figure}[h!]
  \centering
  \begin{SubFloat}{\label{pp:a}Original program}
    \centering
			\begin{lstlisting}[style=sysj,basicstyle=\normalsize\ttfamily,morekeywords={emit,trap,pause,exit,wait_exactly}]
while(true) {
 wait_exactly(1 ms);
 //extra pause
 pause;
 emit S;
}
\end{lstlisting}
\end{SubFloat}
\hspace{9pt}
  \begin{SubFloat}{\label{pp:b}Rewritten program}
			\begin{lstlisting}[style=sysj,basicstyle=\normalsize\ttfamily,morekeywords={emit,trap,pause,exit,delay}]
while(true) {
 while(true) { 
  trap(T) {
   int x = 0;
   while(true) {
    x = x + 1;
    pause;
    if (x == d)
     exit(T);
   }
  }
  //extra pause
  pause;
  emit S;
 }
}
\end{lstlisting}
  \end{SubFloat}
\caption{Extra pause statements in periodic tasks}
  \label{fig:periodic2}
\end{figure}

The system designer needs to be aware that just introducing a wait
construct does not make the reaction (or CD) periodic. The program on
the left does \textit{not} emit signal \texttt{S} every \texttt{1 ms},
because of the extra \texttt{pause} statement. Introducing additional
pauses introduces more logical ticks, consequently increasing the
real-time wait period. Static analysis techniques from real-time
community can be used to automatically determine the wait specification,
in such cases, but describing these techniques is out of the scope of
this work.

\subsection{Need for time analyzable platforms}
\label{sec:need-time-analyzable}

The proposed solution relies on time analyzable platforms, i.e.,
platforms where WCRT and BCRT can be calculated using static analysis
tools. Many might consider this to be too much of a restriction. In this
section we debunk this \textit{perceived} disadvantage.

\subsubsection{Interaction of timers and reactive constructs}
\label{sec:inter-timers-react}

% As mentioned previously, the proposed solution might need
% over-approximation techniques (Algorithm~\ref{alg:2}) to guarantee
% timing requirements (cf. Section~\ref{sec:extend-tehcn-vari}). This
% might give the readers an impression that timer based solution proposed
% for standard real-time operating systems should be utilized for
% GALS/synchronous programs. This is a fallacy. Bourke et
% al~\cite{Bourke2009a} have investigated this technique
% unsuccessfully. We elaborate upon the interplay of timers and reactivity
% to make the problem of using timers explicit.

Preemption plays an important role in reactive languages. One needs to
carefully consider the interplay of wait constructs semantics with the
preemption semantics of reactive languages. Previous attempts at
incorporating postponement using external timers has only been partially
successful, because of the complex interplay between real-time and
preemption. Consider the simple example below, which models real-time
using external timers as in~\cite{rsh94}.

\begin{figure}[h!]
  \centering
	\vspace{-10pt}
		\begin{lstlisting}[style=sysj,basicstyle=\normalsize\ttfamily,morekeywords={emit,trap,pause,exit,delay,suspend}]
suspend(S) {
 emit START_TIMER(10); 
 await (TIMER);emit O1;
}
		\end{lstlisting}
  \caption{Interaction of preemption and external timers}
  \label{fig:preemp}
	\vspace{-10pt}
\end{figure}

As identified in~\cite{Bourke2009a} \texttt{suspend} does not play well
with the external timer. The program in Figure~\ref{fig:preemp} sends a
signal to an external timer and waits for 10 ms to pass by. Consider
what happens when signals \texttt{S} and \texttt{TIMER} occur in the
same logical tick, the \texttt{await} statement is never executed (due
to suspend) and hence, we enter a deadlock. Such problems are completely
avoided in our technique, because we convert the real-time wait
constructs into logical waits (\texttt{pause} constructs), which
interact well with preemption.  Furthermore, synchronous/GALS languages
have always been targeted at real-time analyzable platforms and
constrained
environments~\cite{DBLP:journals/pieee/SifakisTY03,boldt07}. One can use
our technique within such a setting with ease.

\subsubsection{The timer resolution problem}
\label{sec:resolution-real-time}

\begin{figure}[h!]
  \centering
	\vspace{-10pt}
		\begin{lstlisting}[style=sysj,basicstyle=\normalsize\ttfamily,morekeywords={emit,trap,pause,exit,delay,suspend}]
{
 //reaction R1
 emit START_TIMER(1 ms); 
 await (TIMER);emit O1;
}
|| //synchronous parallel
{
 //reaction R2
 // do something
}
		\end{lstlisting}
  \caption{Resolution of external timers}
  \label{fig:resolution}
	\vspace{-10pt}
\end{figure}

There is yet another problem with timer based systems. Let us consider
the program in Figure~\ref{fig:resolution}. On a cursory look this
example should work fine. An external timer is started in reaction R1,
which counts down from 1 ms. Once this time has elapsed, a signal
\texttt{TIMER} is generated from this external timer, which should emit
O1 in turn. There is no \texttt{suspend} construct encapsulating the
timer and hence all seems fine. Let us now consider what happens to the
whole CD, including reaction R2; while reaction R1 is waiting for the
\texttt{TIMER} signal, reaction R2 is making progress. Input signals can
only be captured from the environment at the logical tick boundaries
(cf. Section~\ref{sec:mapping-logical-time}). In
Figure~\ref{fig:resolution}, reaction R2 determines the length of the
tick, because it is performing the heavier computation. Consider what
happens if the external timer generates the \texttt{TIMER} signal, but
the system is still performing computation and the logical tick has not
yet finished -- we \textit{miss} this \texttt{TIMER} signal from the
environment. Thus, even in the absence of the \texttt{suspend}
construct, we are not guaranteed that the program will capture the
external timer signal. In fact, the WCRT needs to be smaller than 1 ms
for the above program to perform as expected. But, this again implies
need for a time analyzable platform even when using external timers.

In general we observe that the WCRT determines the lowest resolution of
any real-time construct whether it be external timer independent, like
the proposed technique, or external timer dependent, like above. This
discovery is one of the key insights for proposing the solutions
described in this paper.

% \subsection{Other semantic discussion}
% \label{sec:other-semant-dissc}

\subsection{Interaction of channel communication and wait constructs}
\label{sec:inter-chann-comm}

Channels, used for communication between reactions in asynchronously
running CDs, are an addition in the SystemJ language. Like interaction
of preemption and waits, conversion of real-time wait constructs to
logical ticks also interacts well with channel rendezvous, because the
semantics of interaction are well defined~\cite{amal10}. More
importantly, we need to consider the interplay of channel communication
and WCRT/BCRT analysis. Since, channel communication does not stop
logical time (see~\cite{amal10}) WCRT/BCRT are unaffected by channel
communication. % The response time to input signals \textit{is} though!
% But, we are unconcerned with the response time, analysis in this paper
% and it remains a future research avenue.

% \subsubsection{Interaction of synchronous parallelism and delay}
% \label{sec:inter-synchr-parall}

% {\color{red}Maybe we should talk about how the control flow in other
%   reactions and CDs is not affected by the delay statement.}



%%% Local Variables: 
%%% mode: latex
%%% TeX-master: "paper"
%%% End: 
