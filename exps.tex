
\section{Experimental results}
\label{sec:experimental-results}
\begin{table}[t]
\centering
\twocolumn[

\begin{@twocolumnfalse}
\caption{Benchmark programs using \texttt{delay} constructs} \label{table:benchmark}
\scalebox{0.85}{
\begin{tabular}{|c| c | c | c | c | c | c | c |}
	\cline{4-8}
	\multicolumn{3}{c|}{}											& HRTCS 		& Motor 		  & Robot 			  & AECS/CD1  		   & AECS/CD2\\ \hline
	\multirow{12}{*}{JOP} 		& \multicolumn{2}{|c|}{BCRT} 		&0.0334 ms 		& 0.0361 ms		  & 0.0325 ms		  & 0.1683 ms		   & 0.1881 ms\\ \cline{2-8}   
								& \multicolumn{2}{|c|}{WCRT} 		&0.1120 ms 		& 0.1116 ms		  & 0.0601 ms		  & 0.7147 ms		   & 1.0115 ms\\ \cline{2-8} 
								& \multirow{2}{*}{Delay 1}	 & M..N &50.3-200.3 ms 	& 2.4-7.4772 ms	  & 1230-2274.0263 ms & 10000-42473.2197 ms& 10000-53772.0045 ms \\ \cline{3-8} 
								&			   & d			 		&1506-1787  	& 67 			  & 37861			  & 59427		       &53160\\ \cline{2-8}
								& \multirow{2}{*}{Delay 2}	 & M..N & 	N/A 		& 1.667-5.2452 ms & N/A				  & N/A 			   &10000-53772.0045 ms\\ \cline{3-8} 
								&			   & d			 		& 	N/A	 		& 47			  & N/A				  & N/A       		   &53160\\ \cline{2-8}
								& \multirow{2}{*}{Delay 3}	 & M..N & 	N/A	 		& 0.05-0.2232 ms  & N/A				  & N/A       		   &N/A\\ \cline{3-8} 
								&			   & d			 		& 	N/A	 		& 2		      	  &	N/A				  & N/A       		   &N/A\\ \cline{2-8}
								& \multirow{2}{*}{Delay 4}	 & M..N & 	N/A	 		& 0.3-1.0044 ms   & N/A				  &	N/A       		   &N/A\\ \cline{3-8} 
								&			   & d				 	& 	N/A	 		& 9				  & N/A				  &	N/A      		   &N/A\\ \cline{2-8}
								& \multirow{2}{*}{Delay 5}	 & M..N & 	N/A	 		& 0.733-2.3436 ms & N/A				  &	N/A      		   &N/A\\ \cline{3-8} 
								&			   & d			 		& 	N/A	 		& 21 			  &	N/A				  & N/A      		   &N/A\\ \hline
	\multirow{12}{*}{TP-JOP} 	& \multicolumn{2}{|c|}{BCRT} 		&0.0028 ms 		& 0.0060 ms		  & 0.0011 ms		  & 0.0532 ms		   & 0.0292 ms\\ \cline{2-8}   
								& \multicolumn{2}{|c|}{WCRT} 		&0.0329 ms 		& 0.0393 ms		  & 0.0361 ms		  & 0.5072 ms		   & 0.8388 ms\\ \cline{2-8} 
								& \multirow{2}{*}{Delay 1}	 & M..N &50.3-600.3 ms 	& 2.4-15.6265 ms  & 1230-42317.8276 ms& 10000-95358.8578 ms& 10000-287757.9834 ms \\ \cline{3-8} 
								&			   & d			 		&18209-18232 	& 398 			  & 1171429			  & 188015		       &343054\\ \cline{2-8}
								& \multirow{2}{*}{Delay 2}	 & M..N & 	N/A 		& 1.667-10.8757 ms& N/A				  & N/A 			   &10000-287757.9834 ms\\ \cline{3-8} 
								&			   & d			 		& 	N/A	 		& 277			  & N/A				  & N/A       		   &343054\\ \cline{2-8}
								& \multirow{2}{*}{Delay 3}	 & M..N & 	N/A	 		& 0.05-0.3534 ms  & N/A				  & N/A       		   &N/A\\ \cline{3-8} 
								&			   & d			 		& 	N/A	 		& 9		      	  &	N/A				  & N/A       		   &N/A\\ \cline{2-8}
								& \multirow{2}{*}{Delay 4}	 & M..N & 	N/A	 		& 0.3-1.9631 ms   & N/A				  &	N/A       		   &N/A\\ \cline{3-8} 
								&			   & d				 	& 	N/A	 		& 50			  & N/A				  &	N/A      		   &N/A\\ \cline{2-8}
								& \multirow{2}{*}{Delay 5}	 & M..N & 	N/A	 		& 0.733-4.79 ms   & N/A				  &	N/A      		   &N/A\\ \cline{3-8} 
								&			   & d			 		& 	N/A	 		& 122 			  &	N/A				  & N/A      		   &N/A\\ \cline{2-8}
	\hline
\end{tabular}
}
\end{@twocolumnfalse}
]

\end{table}
In this section, we present a set of SystemJ benchmark programs in which real-time requirements should be met.
We have carried out a set of experiments to obtain BCRT and WCRT of the benchmark programs on two execution platforms called Java Optimized Processor(JOP) 
\cite{jop:jnl:jsa2007} and TP-JOP \cite{6119095}. JOP is a hardware implementation of the JVM which enables real-time execution of Java programs by translating
Java bytecodes into a sequence of JOP's native instructions called \emph{microcode} which is time-predictable. As SystemJ's default compilation
target is Java source code, JOP is an excellent platform which enables us to analyze timing properties of SystemJ program. On the other hand, there is also
an option where compiled code could be more tightly coupled to specific platforms such as TP-JOP, which execute SystemJ's kernel statements more
efficiently. By utilizing our internal tools in conjunction with Worst-Case-Execution-Time (WCET) analyzer \cite{jop:jnl:jsa2007} 
provided by JOP tools, we were able to estimate BCRT and WCRT of our benchmark programs: (full-name?) HRTCS, Stepper motor controller (Motor) \cite{}, Robot 
and Access Environment Control System (AECS) \cite{}.

%HRTCS, as already illustrated in Section \ref{sec:intr-motiv}, tests for an ability of one's responsiveness by generating green
%lights between the real-time range \emph{M - N}. For AECS \ref{} we have replaced the external timer with our delay constructs. Stepper motor controller
%is originally introduced in \cite{Bourke2009a} which converted into SystemJ.

As one can see in Table \ref{table:benchmark}, BCRT and WCRT of overall programs are smaller when they run on TP-JOP. For example,
WCRT and BCRT of HRTCS are \(\times\)11.9 and \(\times\)3.4 smaller, respectively, for TP-JOP (0.0028, 0.0329 ms) compared to JOP (0.0334, 0.1120 ms). On the other
hand, required logical delays \emph{d} are generally bigger on TP-JOP e.g. \(\times\)4.5 - \(\times\)5.9 and \(\times\)30.9 in 
Motor and Robot respectively. It is expected as their BCRT and WCRT differences are bigger on TP-JOP. 
This also led to greater increase in minimum upper real-time bounds \emph{N} for TP-JOP in every example.
In AECS, there are two clock-domains using delay constructs that each has their own BCRT and WCRT. Again, individual BCRT and WCRT is smaller 
whereas \emph{d} is bigger on TP-JOP. One important property to note here, is that BCRT and WCRT of any clock-domains are invariant of \emph{d} as explained 
in Section \ref{sec:intr-real-time}. It is our fundamental assumption to find \emph{d} in our benchmark programs. 



%%% Local Variables: 
%%% mode: latex
%%% TeX-master: "paper"
%%% End: 














