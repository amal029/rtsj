% THIS IS SIGPROC-SP.TEX - VERSION 3.1
% WORKS WITH V3.2SP OF ACM_PROC_ARTICLE-SP.CLS
% APRIL 2009
%
% It is an example file showing how to use the 'acm_proc_article-sp.cls' V3.2SP
% LaTeX2e document class file for Conference Proceedings submissions.
% ----------------------------------------------------------------------------------------------------------------
% This .tex file (and associated .cls V3.2SP) *DOES NOT* produce:
%       1) The Permission Statement
%       2) The Conference (location) Info information
%       3) The Copyright Line with ACM data
%       4) Page numbering
% ---------------------------------------------------------------------------------------------------------------
% It is an example which *does* use the .bib file (from which the .bbl file
% is produced).
% REMEMBER HOWEVER: After having produced the .bbl file,
% and prior to final submission,
% you need to 'insert'  your .bbl file into your source .tex file so as to provide
% ONE 'self-contained' source file.
%
% Questions regarding SIGS should be sent to
% Adrienne Griscti ---> griscti@acm.org
%
% Questions/suggestions regarding the guidelines, .tex and .cls files, etc. to
% Gerald Murray ---> murray@hq.acm.org
%
% For tracking purposes - this is V3.1SP - APRIL 2009

\documentclass{acm_proc_article-sp}

% correct bad hyphenation here
\hyphenation{op-tical net-works semi-conduc-tor}

\usepackage{courier}
\usepackage{color}
\usepackage{listings}
\usepackage{pgfplots}
\pgfplotsset{compat=1.8}
\usepgfplotslibrary{groupplots}
\usepackage{tikz}
\usetikzlibrary{patterns}
\usepackage{booktabs}
\usepackage{multirow}
\usepackage{url}
%\usepackage{amsmath}
%\usepackage{amsthm}
%\usepackage{amsfonts}
\usepackage[labelfont=bf,textfont=bf]{caption}
\usepackage{subfig}
\usepackage{graphicx}
\usepackage{epstopdf}
\usepackage{paralist}
\usepackage[boxed,linesnumberedhidden]{algorithm2e}
\pagestyle{empty}
\newtheorem{theorem}{Theorem}[section]
\newtheorem{lemma}[theorem]{Lemma}
\newtheorem{proposition}[theorem]{Proposition}
\newtheorem{corollary}[theorem]{Corollary}
\usepackage{mdwlist}
% \usepackage{bibspacing}
\makeatletter
\newbox\sf@box
\newenvironment{SubFloat}[2][]%
{\def\sf@one{#1}%
\def\sf@two{#2}%
\setbox\sf@box\hbox
\bgroup}%
{ \egroup
\ifx\@empty\sf@two\@empty\relax
\def\sf@two{\@empty}
\fi
\ifx\@empty\sf@one\@empty\relax
\subfloat[\sf@two]{\box\sf@box}%
\else
\subfloat[\sf@one][\sf@two]{\box\sf@box}%
\fi}
\makeatother
\lstdefinestyle{sysj}{
  belowcaptionskip=1\baselineskip,
  breaklines=true,
  xleftmargin=\parindent,
  language=Java,
  showstringspaces=false,
  basicstyle=\scriptsize\ttfamily,
  keywordstyle=\bfseries\color{green!40!black},
  commentstyle=\itshape\color{purple!40!black},
  identifierstyle=\color{blue},
  stringstyle=\color{orange},
}

\begin{document}
\newcommand{\hj}[1]{\textcolor{black}{#1}}

\title{Times square -- marriage of real-time and logical-time in GALS
and synchronous languages}
%\subtitle{[Extended Abstract]
%\titlenote{A full version of this paper is available as
%\textit{Author's Guide to Preparing ACM SIG Proceedings Using
%\LaTeX$2_\epsilon$\ and BibTeX} at
%\texttt{www.acm.org/eaddress.htm}}}
%
% You need the command \numberofauthors to handle the 'placement
% and alignment' of the authors beneath the title.
%
% For aesthetic reasons, we recommend 'three authors at a time'
% i.e. three 'name/affiliation blocks' be placed beneath the title.
%
% NOTE: You are NOT restricted in how many 'rows' of
% "name/affiliations" may appear. We just ask that you restrict
% the number of 'columns' to three.
%
% Because of the available 'opening page real-estate'
% we ask you to refrain from putting more than six authors
% (two rows with three columns) beneath the article title.
% More than six makes the first-page appear very cluttered indeed.
%
% Use the \alignauthor commands to handle the names
% and affiliations for an 'aesthetic maximum' of six authors.
% Add names, affiliations, addresses for
% the seventh etc. author(s) as the argument for the
% \additionalauthors command.
% These 'additional authors' will be output/set for you
% without further effort on your part as the last section in
% the body of your article BEFORE References or any Appendices.

\numberofauthors{3} %  in this sample file, there are a *total*
% of EIGHT authors. SIX appear on the 'first-page' (for formatting
% reasons) and the remaining two appear in the \additionalauthors section.
%
\author{
% You can go ahead and credit any number of authors here,
% e.g. one 'row of three' or two rows (consisting of one row of three
% and a second row of one, two or three).
%
% The command \alignauthor (no curly braces needed) should
% precede each author name, affiliation/snail-mail address and
% e-mail address. Additionally, tag each line of
% affiliation/address with \affaddr, and tag the
% e-mail address with \email.
%
% 1st. author
\alignauthor
Ben Trovato\titlenote{Dr.~Trovato insisted his name be first.}\\
       \affaddr{Institute for Clarity in Documentation}\\
       \affaddr{1932 Wallamaloo Lane}\\
       \affaddr{Wallamaloo, New Zealand}\\
       \email{trovato@corporation.com}
% 2nd. author
\alignauthor
G.K.M. Tobin\titlenote{The secretary disavows
any knowledge of this author's actions.}\\
       \affaddr{Institute for Clarity in Documentation}\\
       \affaddr{P.O. Box 1212}\\
       \affaddr{Dublin, Ohio 43017-6221}\\
       \email{webmaster@marysville-ohio.com}
% 3rd. author
\alignauthor Lars Th{\o}rv{\"a}ld\titlenote{This author is the
one who did all the really hard work.}\\
       \affaddr{The Th{\o}rv{\"a}ld Group}\\
       \affaddr{1 Th{\o}rv{\"a}ld Circle}\\
       \affaddr{Hekla, Iceland}\\
       \email{larst@affiliation.org}
%\and  % use '\and' if you need 'another row' of author names
}
\date{30 July 1999}

\maketitle
\begin{abstract}

	In this paper we introduce exact and non-exact real-time waits in
	reactive Globally Asynchronous Locally Synchronous (GALS) programming
	languages and synchronous languages as their subset. The language
	constructs that allow use of real-time waits are illustrated on the
	SystemJ GALS language. They allow system designers to explicitly use,
	at the specification level, not only logical time but also the
	real-time in order to control \hj{program execution}. The introduced
	concepts utilize execution platforms that allow finding best and worst
	reaction time of a GALS or synchronous program.

\end{abstract}

% A category with the (minimum) three required fields
\category{H.4}{Information Systems Applications}{Miscellaneous}
%A category including the fourth, optional field follows...
\category{D.2.8}{Software Engineering}{Metrics}[complexity measures, performance measures]

\terms{Theory}

\keywords{ACM proceedings, \LaTeX, text tagging} % NOT required for Proceedings

%\setlength\intextsep{2cm}
\setlength\abovecaptionskip{10pt}

\section{Introduction and motivation}
\label{sec:intr-motiv}

Reactive languages~\cite{gber931,amal10} are a class of programming
languages used for designing and implementing reactive systems, which
continuously respond to input from their environment. These languages
have been successfully used in programming a plethora of systems such as
fly-by-wire in Airbus~\cite{eairbus}, security surveillance
systems~\cite{amal121}, etc. Usually these systems also need to meet
real-time constraints imposed by the environment. Yet, these languages
do not support describing real-time statements as first class language
constructs.  For example, one cannot describe a simple real-time delay
(postponement) of an operation for 0.2 ms. These languages are based on
formal semantics, essential for correct by construction code generation
and formally verifying functional properties of developed programs, but
leave nonfunctional properties such as timing behavior as an
implementation detail~\cite{boldt07}. Arguably, rightly so, because
physical time cannot be incorporated without information about the
underlying execution platform.  But, one can still reason in terms of
logical time. These languages support discrete logical clock rather than
a discrete physical (real-time) clock. The real-time period of the
discrete logical clock (also referred to as a logical tick) is not fixed
and it is determined by the speed of computation of the program for
different input signals. Unlike a discrete physical clock, which has
fixed real-time period, the period of the logical clock is elastic.
Although the timing model with logical clock has worked well for
reactive synchronous/GALS languages in designing discrete systems, there
is a need to introduce real-time since often, implementation models are
in the continuous or discrete real-time
domain~\cite{DBLP:journals/pieee/SifakisTY03}. This paper targets
enhancing formal GALS and synchronous languages, by adding real-time
constructs. The real-time constructs speed up a design process by
allowing system designers to explicitly program real-time postponements
within the language itself without need of external resources such as
timers.

%Moreover, adding timing
%capabilities as first class language constructs create additional
%opportunity for formally verifying functional and non-functional
%requirements before system deployment.

\subsection{Motivation via examples of real-time systems programmed in a
reactive language SystemJ}
\label{sec:motivating-example}

% system{
%  interface{
%   input signal TOUCH;
%   output signal GREEN_LIGHT;
%   output char channel S;
%   input char channel R;
%  }

Real-time operating systems~\cite{barry2009using} usually provide two
types of mechanisms for introducing real-time in the developed program:
(1) the ability to perform a timeout and (2) the ability to run tasks
periodically with some real-time period $T$. We would like to provide
similar mechanisms in reactive languages for real-time program
development. In fact, we provide \textit{exact} and \textit{non exact}
real-time control mechanisms -- a generalization of the two real-time
mechanisms introduced in real-time operating systems.

\subsubsection{Programming using the non exact real-time control
  construct; the \textrm{\texttt{wait\_inbetween}} statement}
\label{sec:progr-using-non}

\begin{figure}[t!]
	\vspace{-10pt}
        \begin{SubFloat}{\label{delay:a}A human responsiveness
            system (HRTCS)}
        \begin{lstlisting}[style=sysj,morekeywords={sustain,send,receive,abort,await,emit,present,trap,pause,exit,wait_inbetween,wait_exact,suspend}]
{ // Clock-domain 1
while(true) {
  trap(T){
   // Abort if user touches the screen
   abort(TOUCH){
    {sustain GREEN_LIGHT;} //reaction R1
    || // synchronous parallel operator
    {
     //exit after any where from 50.3 to 200.3 ms.
     wait_inbetween (50.3 .. 200.3 ms); 
     exit(T); 
    } // reaction R2
   } do {send S(1);} // Send 1 if touched 
  } do {send S(0);} // Else Send 0
  pause; pause; pause; // pass some time before restarting
 }
}
>< // asynchronous operator
{ // Clock-domain 2
  while(true){ receive S; 
      if(#S == 1) counter++; else counter--;}
}
\end{lstlisting}
\end{SubFloat}
\begin{SubFloat}{\label{d:b}Timing diagram -- the X-axis shows the
    execution of the \texttt{wait\_inbetween} statement. The Y-axis is
    the time elapsed. The arrow shows one example of the expiration of
    the \mbox{\texttt{wait\_inbetween (50.3..200.3 ms)}} statement.}
  % \begin{figure}[h!]
    \centering
    \includegraphics[scale=0.7]{FF1}
  % \end{figure}
\end{SubFloat}
\caption{Programming a human responsiveness system (HRTCS) with
  \texttt{wait\_inbetween}}
% }
% \end{center}
\label{fig:1}
\end{figure}

The \texttt{wait\_inbetween (M..N)} statement postpones the program
control flow from proceeding to the following statement for a minimum of
$M$ time units and a maximum of $N$ time units, as shown in
Figure~\ref{d:b}.

Consider a machine used to test human response time and collect this
data for future research purposes. The machine switches on a green light
on a touch screen for anywhere \textit{in between} 50.3 ms to 200.3 ms,
if the user can touch this green light then she is successful and \hj{an
  integer \texttt{1} is sent to another process in order to increase an
  integer counter, if unsuccessful a \texttt{0} is sent instead to
  decrease the counter}. Different statistics and analyses on human
responsiveness can be based on the collected data.

The SystemJ pseudo-code~\cite{amal10} for such a system, along with the
expected timing behavior of the \texttt{wait\_inbetween} statement, is
shown in Figure~\ref{fig:1}. We use the SystemJ language in this paper,
because being a \textit{Globally Asynchronous Locally Synchronous}
(GALS) language it helps us easily describe a larger class of systems
(like HRTCS in Figure~\ref{delay:a}) compared to purely synchronous
languages like Esterel. Moreover, being a super-set of Esterel, SystemJ
allows us to explore the marriage of real-time and logical time not only
in the GALS, but also in the synchronous setting.

The SystemJ program in Figure~\ref{delay:a} is divided into two
\hj{mutually asynchronous} clock-domains, the first clock-domain
continuously displays a green light (\texttt{GREEN\_LIGHT} signal) on
the touch screen sensor and waits for the user to respond. If the user
is able to touch the screen within the specified time of at least 50.3
ms and at most 200.3 ms, a positive response is sent to the second
clock-domain. All the SystemJ programming constructs used to implement
this system are described in Table~\ref{tab:1} \hj{and more details can
  be found in~\cite{amal10}}. The \texttt{wait\_inbetween}
specification, which models the non exact postponement between 50.3 ms
and 200.3 ms is not supported in the current language and cannot be
compiled with the current SystemJ compiler. We introduce such
\texttt{wait\_inbetween} statements in SystemJ. Such non exact real-time
mechanisms are useful for developing systems where the real-time
postponement is not known or should not be known a priori.

\subsubsection{Programming a timeout using the non exact real-time
  construct; the \texttt{wait\_atleast} statement}
\label{sec:progr-time-using}

\begin{figure}[b!]
	\centering
	\vspace{-10pt}
        \begin{SubFloat}{\label{dd:a}Timeout waiting for the input
            signal \texttt{DoorOpened}}
        \begin{lstlisting}[style=sysj,morekeywords={abort,await,emit,present,trap,pause,exit,wait_atleast,suspend}]
trap(T){
  abort(DoorOpened){
    wait_atleast(10000 ms);
    exit(T);  
  }
}
\end{lstlisting}
\end{SubFloat}
\begin{SubFloat}{\label{dd:b}Timing diagram -- the X-axis shows the
    execution of the \texttt{wait\_atleast} statement. The Y-axis is the
    time elapsed. The arrow shows one possible expiration of
    \texttt{wait\_atleast (10000 ms)} statement.}
\includegraphics[scale=0.7]{FF2}
\end{SubFloat}
\caption{Programming a timeout with \texttt{wait\_atleast}}
\label{dd}
\end{figure}

The \texttt{wait\_atleast (M)} statement is used to postpone the program
control flow from proceeding to the next statment for a minimum of $M$
time units. The maximum postponement is unbounded, but countable
(Figure~\ref{dd:b}).

There are many instances when one would like to wait on an input from
environment for only a specified amount of time. If the signal is not
received, then a timeout is generated so that the program can make
progress. Such a timeout programmed in the SystemJ pseudo-code along
with its timing behavior is shown in Figure~\ref{dd}. The SystemJ
program is waiting on an input signal, \texttt{DoorOpened}, from the
environment for 10 seconds and makes progress irrespective of the
reception of this signal after 10 seconds.

The timing diagram, Figure~\ref{dd:b}, shows that the
\texttt{wait\_atleast} statement waits for a \textit{at least} 10
seconds, but unlike \texttt{wait\_inbetween} there is no upper bound.
For example in Figure~\ref{dd:b} program can proceed to the next
statement at 11 seconds. The statement following the
\texttt{wait\_atleast} statement is executed after a finite countable,
but unbounded postponement. Such a statement is useful for programming
low priority tasks.


\subsubsection{Programming periodic tasks using the exact real-time
  construct; the \texttt{wait\_exact} statement}
\label{sec:progr-using-exact}

The previous wait statements are non exact, i.e., the \hj{time of}
execution of the statement following the wait statement cannot be
controlled exactly. The final variant we introduce allows doing exactly
that. We call it \texttt{wait\_exact} and this statement can be used to
program exact timeouts, periodic tasks, etc.

\begin{figure}[t!]
  \centering
	\vspace{-10pt}
        \begin{SubFloat}{\label{pp:a}Periodically emitting signal S}
        \begin{lstlisting}[style=sysj,morekeywords={emit,trap,pause,exit,wait_exact}]
while(true) { 
 wait_exact (1 ms); 
 emit S; 
 //do something 
}
\end{lstlisting}
\end{SubFloat}

\begin{SubFloat}{\label{p1:b}Timing diagram -- the X-axis shows the
    execution of the \texttt{wait\_exact} statement. The Y-axis is the
    time elapsed. The arrow shows the \textit{only} possible expiration
    of the \texttt{wait\_exact (1 ms)} statement.}
  \includegraphics[scale=0.7]{FF3}
\end{SubFloat}
  \caption{Programming a periodic task with wait\_exact}
  \label{fig:p1}
  \vspace{-10pt}
\end{figure}

One such example; periodic emission of a signal to the environment is
shown in Figure~\ref{fig:p1}. As before, upon execution of
\texttt{wait\_exact (1 ms)} statement, the time starts elapsing. But,
unlike in previous cases the following statement needs to be executed
exactly after 1 ms has passed, as shown in Figure~\ref{p1:b}.

%%% Local Variables: 
%%% mode: latex
%%% TeX-master: "paper"
%%% End: 

%  LocalWords:  statment


The major motivations for introducing these real-time \textit{wait}
mechanisms and at the same time \textbf{contributions} of the paper are:
\hj{(1) real-time wait does not affect the model of logical time in
	SystemJ program as long as the amount of delay is within boundaries
that can be determined statically by program analysis.} (2) the waiting
model relies on the use of relative instead of absolute real-time, i.e.,
a wait in selected time units is counted from the beginning of the wait
statement and (3) these waiting mechanisms allow mixing behaviors with
real-time features and others with only logical time. All these as well
as semantics of the wait constructs are discussed and illustrated in the
rest of the paper. Our contributions can be further refined as follows:

\begin{enumerate*}
\item Real-time is specified in the $\mathbb{Q}^{>0}$ domain of numbers,
  i.e., non-negative rational numbers.
% \item We consider the property of \textit{non
%     maximal-parallelism}. \textit{Non-maximal parallelism:} There are
%   not always sufficient resources for processes to execute, so
%   effectively processes are allocated and scheduled with different
%   optimization criteria -- reducing computation time, reducing power
%   consumption, etc. This allocation and scheduling is an integral part
%   of any language compilation and needs to be considered when
%   introducing real-time waits.
\item We are the first, to our knowledge, to allow specification of
  both; exact and non-exact real-time in GALS and synchronous programs.
\item We shed light upon a key insight relating real-time and
  logical-time -- \textit{the resolution of any real-time statement is
    dependent on the worst case logical tick time} -- an aspect that
  seems to have escaped in the current literature on incorporation of
  real-time in synchronous/GALS languages.
\item We \textbf{assume} that real-time analyzable platforms are
	available on which SystemJ programs are executed.
\end{enumerate*}

The rest of the paper is arranged as follows:
Section~\ref{sec:background} gives the background on SystemJ and related
techniques required for the rest of the
paper. Section~\ref{sec:intr-real-time} introduces real-time mechanisms
in the GALS paradigm and describes their compilation into logical
time. Section~\ref{sec:proof-correctness} sketches the correctness
proofs for the introduced
algorithms. Section~\ref{sec:experimental-results} gives the
experimental results for a set of real-time applications. In
Section~\ref{sec:disc-perc-limit} we discuss the advantages and
limitations of the proposed approach. Section~\ref{sec:related-work}
positions this work in relation to other approaches to introducing
real-time in other languages. Finally, we conclude in
Section~\ref{sec:concl-future-work}.


%%% Local Variables: 
%%% mode: latex
%%% TeX-master: "paper"
%%% End: 

\section{Background}
\label{sec:background}

\subsection{Brief introduction to SystemJ syntax, semantics and model of
  computation}
\label{sec:brief-intr-syst}

\begin{table}[tb]
\centering
\caption{SystemJ kernel statements and their meaning}
\begin{minipage}{8cm}
  \begin{small}
   \begin{tabular}{|c|p{80pt}|}
     \hline                                                                                     
     \textbf{Kernel Statements} & \textbf{Meaning}\\                                            
     \hline                                                                                     
     \hline                                                                                     
     [\textbf{\texttt{input}}] [\textbf{\texttt{output}}]
     [\textbf{\texttt{type}}] \textbf{\texttt{signal}} S & declare signal S\\                                      
     \hline                                                                                     
     \textbf{\texttt{emit}} S [(value)] & broadcast signal S\\                                                    
     \hline                                                                                     
     \textbf{\texttt{present}} (S) \{p\} else \{q\}& do p if S is present, else do q\\                            
     \hline                                                                                     
     \textbf{\texttt{abort}} (S) \{p\} & preempt program p if S is present\\                                      
     \hline                                                                                     
     \textbf{\texttt{suspend}} (S)\{p\} & suspend p if S is present\\                                             
     \hline                                                                                     
     \textbf{\texttt{trap}} (T)\{p\ldots \textbf{\texttt{exit}} T\ldots\} & preempt p if exit is executed\\                         
     \hline                                                                                     
     p\textbf{\texttt{$||$}}q & run p and q in lock-step\\                                                        
     \hline                                                                                     
     p$><$q & run p and q asynchronously\\                                                      
     \hline                                                                                     
     \textbf{\texttt{send}} C([value]) & send a value through C, blocking
     send\\                                                 
     \hline                                                                                     
     \textbf{\texttt{receive}} C() & receive a value through C, blocking
     receive\\
     \hline                                                                                     
     \textbf{\texttt{pause}} & finish a tick and communicate
     with environment\\
     \hline                                                                                     
   \end{tabular}
  \end{small}
   % \footnotetext[1]{\scriptsize Uppercase}
 \end{minipage}
 \label{tab:1}
\end{table}

Table~\ref{tab:1} presents the SystemJ kernel statements used to program
the control flow. The data-flow is programmed in Java. A number of
derived statements also exist (e.g., \texttt{sustain}, \texttt{await},
etc) that make programming reactive systems easier. The \textit{Model of
  Computation} (MoC) of a simple SystemJ is shown in Figure~\ref{fig:2}.

\begin{figure*}[bth]
\centering
\begin{SubFloat}{\label{fig:2a}Simple SystemJ program}%...verbatim subfigure
\begin{minipage}[b]{0.3\linewidth}% a minipage to control the width...
\begin{verbatim}
while(true){
 abort(A){
  while(true)
   pause;
  emit O;
  // do java computation
}
\end{verbatim}%
\end{minipage}%
\end{SubFloat}
\hspace{1.5cm}%
\begin{SubFloat}[Black box]{\label{fig:2b}Ticks in SystemJ}%
\includegraphics[scale=0.5]{moc}
\end{SubFloat}%
\caption{Simple SystemJ example and corresponding MoC}
\label{fig:2}
\end{figure*}

SystemJ consists of entities called clock-domains (CD)s, each running at
their individual pace. Each CD adheres to the \textit{perfect synchrony}
hypothesis, i.e., all statements execute instantaneously in zero
time. Only the \texttt{pause} statement consumes time, just like in
Esterel~\cite{gber931}. Consider the SystemJ program in
Figure~\ref{fig:2a}, it is waiting for an input signal producing logical
ticks (see Figure~\ref{fig:2b}). Once \texttt{A} is received at tick 4,
output \texttt{O} is emitted to the environment instantaneously.

\subsection{Mapping logical time to physical time}
\label{sec:mapping-logical-time}

The perfect synchrony hypothesis is ideal for programming the
synchronous sub-set of SystemJ. But, the real-world is not that
forgiving, execution of every \textit{micro-step} in the logical zero
time, requires $\delta$ physical time. Time for a complete reaction to
one or more input signal can thus be summarized as $\Delta = \Sigma
(\delta)$. We call $\Delta$ the reaction time. Obviously depending upon
the amount of computation required, $\Delta$ can vary, hence, in order
to satisfy the implicit restriction posed by the synchrony hypothesis --
no input event can be missed, one needs to calculate the \textit{Worst
  Case Reaction Time} (WCRT) and the resultant WCRT needs to be smaller
than the speed of the fastest input event, else the synchrony hypothesis
is violated. A substantial amount of research for calculating the WCRT
of synchronous programs~\cite{proop10,boldt07,wilhelm08} exists. We
refer the reader to any of these research articles, since WCRT analysis
is not the focus of this paper. The opposite of the WCRT is the
\textit{Best Case Reaction Time} (BCRT), we denote the WCRT and BCRT in
Figure~\ref{fig:2b} using \texttt{W} and \texttt{B}, respectively.

We now present two lemmas with proof sketch (due to lack of space) that
we require for the rest of the paper.

\begin{lemma}
  WCRT and BCRT analysis are invariant to values of conditional
  expressions.
\end{lemma}

\begin{proof}
  The physical time ($\delta$) taken by a conditional expression is at
  best affected by the type rather than the value of the
  conditional. Example, comparing the value of 2 floats might take
  longer than comparing the value of 2 integer types on some given
  platform. But, the time taken by the comparison instruction is
  invariant to the value itself.
\end{proof}

\begin{lemma}
  WCRT and BCRT analysis are invariant to channel communication.
\end{lemma}
\begin{proof}
  LATER ....
\end{proof}


%%% Local Variables: 
%%% mode: latex
%%% TeX-master: "paper"
%%% End: 

\section{Introducing real time in SystemJ}
\label{sec:intr-real-time}

We introduce a single \textit{derived} construct called
\mbox{\texttt{delay [M..N]}} in the SystemJ language for real time
control. The resolution of the delay statement is of secondary concern
and is dependent upon the execution platform, without loss of generality
through out this paper we consider the resolution to be always in milli
seconds.

\subsection{Semantics of the delay statement}
\label{sec:semant-delay-stat}

Given a SystemJ program: \texttt{delay [M..N]; p}, where $M \in
\mathbb{Q}^{>0}$ and $N \in \mathbb{Q}^{>0}$, statement $p$ is executed
after real-time $\tau$, such that, $M \leq \tau \leq N$.

We also introduce two variants of the derived statement \texttt{delay}.
\begin{enumerate*}
\item Given a SystemJ program \texttt{delay M; p}, where $M \in
  \mathbb{Q}^{>0}$, statement $p$ is executed after real-time $\tau$,
  such that, $M \leq \tau \leq \infty$. It is important to note that the
  lower bound of the delay construct $M$ is \textit{not} an exact delay,
  but rather the control is allowed to proceed to the next statement
  anytime after the delay of time $M$.
\item Give a SystemJ program \texttt{delay [M..M]; p}, where $M \in
  \mathbb{Q}^{>0}$, statement $p$ is executed after real-time $\tau$,
  such that, $M \leq \tau \leq M$. In this variant, the delay is exact.
\end{enumerate*}

The aforementioned variants are specializations of the general case
\texttt{delay [M..N]}.

\subsection{Rewriting the \texttt{delay} statement}
\label{sec:rewr-delay-stat}

The introduced \texttt{delay} construct is not a kernel construct, but a
\textit{derived} construct built from the kernel constructs
(Table~\ref{tab:1}) in SystemJ. Figure~\ref{fig:3} gives the rewrite of
the \texttt{delay} construct to kernel statements.

\begin{figure}[tb]
    \begin{minipage}{\textwidth}
      \begin{scriptsize}
\begin{verbatim}
trap(T){
 int x = 0;
 while(true){
  x = x + 1;
  if(x == d) exit (T); //wait for "d" ticks
  pause;
 }
}
\end{verbatim}
      \end{scriptsize}
    \end{minipage}
    \caption{The rewrite of \texttt{delay} construct}
    \label{fig:3}
\end{figure}

The fundamental observation is that real-time is converted into logical
time via the \texttt{pause} construct. The rewrite basically maps the
physical notion of time back to the elegant logical notion of time. The
rewrite \textit{delays} a certain number of logical ticks, before
proceeding to the next statement. The number of logical ticks
\texttt{``d''} to \textit{delay} (Figure~\ref{fig:3}) is determined by
the compiler statically at compile time. The value of \texttt{d} is
intricately tied with the WCRT and BCRT of the program and hence the
execution platform.

\subsection{Finding the logical delay \texttt{d}}
\label{sec:find-logic-delay}

\begin{algorithm}[t!]
  \begin{minipage}{1.0\linewidth}
    \SetAlgoLined
    \KwData{WCRT, BCRT, $M \in \mathbb{Q}^{>0}$, $N \in \mathbb{Q}^{>0}$}
    \KwResult{d}
    let $l_1 \leftarrow \lceil \frac{M}{WCRT} \rceil$\;
    let $l_2 \leftarrow \lceil \frac{M}{BCRT} \rceil$\;
    let $u_1 \leftarrow \lfloor \frac{N}{WCRT} \rfloor$\;
    let $u_2 \leftarrow \lfloor \frac{N}{BCRT} \rfloor$\;
    let $F:(l_1,u_1) \rightarrow S_1$\;
    let $F:(l_2,u_2) \rightarrow S_2$\;
    let $D \leftarrow S_1 \cap S_2$\;
    \Return (some $d \in D$)\;
    \caption{Finding the value of \texttt{d}}
    \label{alg:1}
  \end{minipage}
\end{algorithm}

The computation of \texttt{d} is shown in Algorithm~\ref{alg:1}. This
algorithm is carried out for each CD (in SystemJ) or a synchronous
program individually. \textit{The fundamental observation is -- the
  reaction time for each logical tick is elastic -- varying only between
  the BCRT and the WCRT, thus any logical number of ticks \texttt{d}
  that map to the required real-time \texttt{delay} should be chosen in
  such a way that they are invariant to this elasticity.}

Our algorithm takes as input: WCRT, BCRT, and the lower and upper bounds
$M$ and $N$, respectively of the \texttt{delay} construct. We first
divide $M$ and $N$ with the BCRT and WCRT, respectively. This division
gives us the number of individual logical ticks required to delay the CD
(or synchronous program) by the real-time specification. We always
$ceil$ when dividing $M$ and $floor$ when dividing $N$ to make sure that
the resultant values are integers (in domain $\mathbb{N}^{>0}$) and
these functions guarantee that the resultant logical ticks result in
real-time delays between the required range $M-N$. Next, a linear
function $F$ maps these calculated values to a set of equidistant
integer points (values) separated by a unit value -- these points
represent all the logical ticks running at the WCRT and the BCRT,
respectively that satisfy the real-time delay requirements. The
intersection of these two sets gives all the logical ticks that satisfy
the real-time requirements invariant of the reaction time and its
elasticity.

Let us revisit our motivating example to elucidate the algorithm. From
Figure~\ref{fig:1} we know that $M$ is 50ms and $N$ is 100ms,
respectively. Let the WCRT and BCRT be: 13.332ms and 10ms,
respectively. Thus, the algorithm proceeds as follows:

\begin{enumerate*}
\item $l_1 \leftarrow \lceil 50/13.332 \rceil$ and $u_1 \leftarrow
  \lfloor 100/13.332 \rfloor$. $l_1 = 4$ and $u_1 = 7$. We first
  calculate the logical ticks that are always running at the WCRT
  satisfy the required real-time delay.
\item $l_2 \leftarrow \lceil 50/10 \rceil$ and $u_2 \leftarrow
  \lfloor 100/10 \rfloor$. $l_1 = 5$ and $u_1 = 10$. We do the same for
  the BCRT case.
\item $S_1 = \{4,5,6,7\}$ and $S_2 =\{5,6,7,8,9,10\}$. We then map the
  resultant bounds to linear points. Sets $S_1$ and $S_2$ represent
  logical ticks that running at the WCRT and BCRT, respectively always
  satisfy the required real-time constraints.
\item $D = S_1 \cap S_2$, $D = \{5,6,7\}$. Finally, the intersection of
  the two sets gives the set $D$ from which we can pick up any value for
  \texttt{d}.
\end{enumerate*}

The resultant value for \texttt{d} gives the number of logical ticks,
which can run at any physical clock-speed, bounded by the BCRT and the
WCRT of the program and still result in the required real-time delay. We
think this is an elegant solution, because the technique provides
hard-real time guarantee while preserving the essence of synchronous and
GALS programming prescribed by SystemJ and Esterel style
languages. Moreover, this technique considers non maximal-parallelism,
i.e., the delays in logical ticks is calculated after scheduling has
been performed. To our knowledge we are the first to do so.

\subsection{Extending the technique to variants of \texttt{delay}}
\label{sec:extend-tehcn-vari}

\paragraph{The \texttt{delay(M)} construct}
\label{sec:extend-techn-vari}

The \texttt{delay(M)} variant is easily accommodated in the
technique. All one needs to do is find the set $S_1$ and choose a value
from this set.

\paragraph{The \texttt{delay(M..M)} construct}
\label{sec:extend-techn-vari}

The \texttt{delay(M..M)} variant is a little more interesting. Like
before we find sets $S_1$ and $S_2$, and find the intersection of the
two sets to get the value of \texttt{d}. It is possible (and often
likely, as suggested by our experiments in
Section~\ref{sec:experimental-results}) that the resultant set $D$ is
empty (also possible in the case of \texttt{delay(M..N)}, but never
possible in case of \texttt{delay(M)}). In such a case, we automatically
\textit{relax} the upper bound of the \texttt{delay} statement.


\paragraph{Relaxation of the upper real-time bound}
\label{sec:over-appr-relax}

The relaxation algorithm is shown in Algorithm~\ref{alg:2}.
  
\begin{algorithm}[t!]
  \begin{minipage}{1.0\linewidth}
    \SetAlgoLined
    \KwData{$S_2$, $D$, WCRT}
    \KwResult{d}
    \If {$D = \emptyset$} {
      let $j_{0}$ be the first element of set $S_2$, s.t., $|S_2|=Q$\;
      \ShowLn let $N \leftarrow WCRT \times j_0$\;
      let $d \leftarrow j_0$\;
    }
    \Return d\;
    \caption{Calculating the minimum relaxation of the upper real-time
      bound}
    \label{alg:2}
  \end{minipage}
\end{algorithm}


This algorithm results in the smallest relaxation required for the
real-time delay to be satisfied. The algorithm takes as input the WCRT
and set $S_2$, recall that set $S_2$ represents the logical ticks
required to satisfy the real-time requirement at the BCRT. We take the
very first value from set $S_2$ and multiply it with the WCRT to get the
relaxation $N$. The first element of set $S_2$ is returned as the
logical tick delay \texttt{d}.

The fundamental observation is that we have to delay for a minimum of
$M$ units of real-time, hence, under-approximation is out of
question. We can still over-approximate, but to reduce the resultant
error, we should over-approximate by least possible value, which is the
lower bound of set $S_2$. Thus, the lower bound is considered to be the
only element shared between the two sets $S_1$ and $S_2$ and
accordingly, the upper real-time bound is relaxed by the multiplication
of WCRT and the first element of set $S_2$.


\subsection{Programming using the delay construct}
\label{sec:progr-using-delay}

In this section we provide a number of examples to show the different
types of real-time programming paradigms that can be incorporated into
the GALS (and its sub-set synchronous) programming model.

\subsubsection{Non-deterministic time}
\label{sec:non-determ-time}

A number of programs require non-deterministic timing constructs. One of
them is our motivating example -- the human response time
system. Another is a printer-spooler example borrowed from timed
CSP~\cite{Schneider:1999:CRT:555233}. The spooler and printer need to
synchronize using channels. The printer might be unable to print
depending upon the paper tray, similarly, spooler might take sometime to
send the job depending upon its size. Such real-time constraints can be
modeled in SystemJ as below:

\begin{scriptsize}
  
\begin{verbatim}
{await (job); delay([2..10]ms);  //SPOOLER CD
 send mid(job) delay (1ms);}
><
{receive mid; delay([1..30]ms); //PRINTER CD
 emit print(#mid);}
\end{verbatim}
\end{scriptsize}

\subsubsection{Timeout}
\label{sec:timeout}

There are many instances when one would like to wait on an input from
environment for only a specified amount of time. This can be programmed
as:
    \begin{scriptsize}
\begin{verbatim}
// timeout after 1ms.
trap(T){
 {await(A);}||{delay(1ms);exit(T); }
}
\end{verbatim}
    \end{scriptsize}

\subsubsection{Periodic reactions}
\label{sec:periodic-reactions}

A reaction, or a whole CD, can be programmed to run periodically like
so:

\begin{scriptsize}
  
\begin{verbatim}
{
delay (1ms); emit S; //do something
}
\end{verbatim}
\end{scriptsize}
A periodic reaction (or CD) requires special consideration. Since the
delay statement is converted into \texttt{pause} constructs. One should
not introduce extra \texttt{pause} constructs when building periodic
reactions (or CDs). This is essential since, introducing pauses would
introduce more logical ticks.

\subsubsection{Interaction of preemption and delays}
\label{sec:inter-preempt-delays}

Preemption plays an important role in reactive languages. One needs to
carefully consider the interplay of \texttt{delay} semantics with the
preemption semantics of reactive languages. Other attempts at
incorporating delays (using external timers) have only been partially
successful, because of the complex interplay between real-time and
preemption. Consider the simple example below, which models real-time
using external timers as in~\cite{rsh94}.

\begin{scriptsize}
  
\begin{verbatim}
suspend{S} {
 emit START_TIMER(10); await (TIMER); 
 emit O1;
}
\end{verbatim}
\end{scriptsize}

As identified in~\cite{Bourke2009a} the \texttt{suspend} does not play
well with the external timer. The above program sends a signal to an
external timer and waits for 10ms to pass by. Consider what happens when
signals \texttt{S} and \texttt{TIMER} occur together, the \texttt{await}
statement is never executed and hence, we enter a deadlock. Such
problems are completely avoided in our technique, because we convert the
real-time delays into logical delays (\texttt{pause} constructs), which
bode well with preemption.

\subsubsection{Interaction of channel communication and delays}
\label{sec:inter-chann-comm}

Channels are an addition in the SystemJ language. Like interaction of
preemption and delays, conversion of rela-time delays to logical ticks
also bodes well with channel rendezvous, because the semantics of
interaction are well defined~\cite{amal10}. More importantly, we need to
consider the interplay of channel communication and WCRT/BCRT
analysis. Since, channel communication does not stop logical time
(see~\cite{amal10}) WCRT/BCRT are unaffected by channel
communication. The response time to input signals \textit{is} though!
But, we are unconcerned with the response time, analysis in this paper
and it remains a future research avenue.

%%% Local Variables: 
%%% mode: latex
%%% TeX-master: "paper"
%%% End: 


\section{Proofs of correctness, completeness, and soundness}
\label{sec:proof-correctness}

Section~\ref{sec:intr-real-time} describes the techniques and algorithms
for introducing various types of wait statements providing real-time
control in the SystemJ language. In this section we give the formal
proof of correctness for the algorithms described in
Section~\ref{sec:intr-real-time}.

\begin{lemma}
  Algorithm~\ref{alg:1} gives a value for \texttt{d} such that: $\lceil
  \frac{M}{BCRT} \rceil \leq d \leq \lfloor \frac{N}{WCRT} \rfloor$
  provided $D$ is a non-empty set for construct \texttt{wait\_inbetween
    (M..N)}
\label{lemma:1}
\end{lemma}
\begin{proof}
  This lemma is trivially true from the definition of set $D$ in
  Algorithm~\ref{alg:1}.
\end{proof}

\begin{theorem}
  Given \texttt{wait\_inbetween (M..N)} Algorithm~\ref{alg:1} gives a
  value \texttt{d}, provided $D$ is a non empty set, such that for any
  given reaction time $\hj{\Delta}$ \hj{in the range of} $BCRT \leq
  \Delta \leq WCRT$, $M \leq \Delta \times d \leq N$ holds.
\label{th:1}
\end{theorem}
\begin{proof}
  We use case analysis and contradiction to prove the above theorem.
  \begin{compactenum}[\hspace{0.25cm} 1.]
  \item Proof for case $\Delta = WCRT$: Assume \mbox{$d \times WCRT <
      M$}, then $d < \frac{M}{WCRT}$. $BCRT \leq WCRT$, by definition,
    hence, $\frac{M}{WCRT} \leq \frac{M}{BCRT}$. Thus for the assumption
    to hold, $d < \frac{M}{BCRT}$, which contradicts
    Lemma~\ref{lemma:1}. The assumption $d \times WCRT >N$ is trivially
    proven from Lemma~\ref{lemma:1}.
  \item Proof for case $\Delta = BCRT$: Like above, the assumption $d
    \times BCRT < M$ is trivially proven from Lemma~\ref{lemma:1}. For
    $d \times BCRT > N$ to hold, $d > \frac{N}{BCRT}$ should hold. But,
    $BCRT \leq WCRT$, by definition, hence, $\frac{N}{BCRT} \geq
    \frac{N}{WCRT}$, which again contradicts Lemma~\ref{lemma:1}.
  \item Proof for case $BCRT < \Delta < WCRT$: For $d \times \Delta < M$
    to hold, $d < \frac{M}{\Delta}$ should hold. By definition, $\Delta
    > BCRT$, thus, $\frac{M}{BCRT} < \frac{M}{\Delta}$, hence, $d <
    {M}{BCRT}$ should hold, which contradicts Lemma~\ref{lemma:1}. We
    can prove the other case $d \times \Delta \leq N$, using the upper
    bound from Lemma~\ref{lemma:1} similarly.
  \end{compactenum}
\end{proof}

Next, we prove the completeness property.
\begin{theorem}
  Given \texttt{wait\_inbetween (M..N)} Algorithm~\ref{alg:1} gives a
  value \texttt{d}, provided $D$ is a non empty set, such that for
  \textbf{all} given reaction times $\Delta$ in the range of $BCRT \leq
  \Delta \leq WCRT$, $M \leq \Delta \times d \leq N$ holds.
\end{theorem}
\begin{proof}
  From Theorem~\ref{th:1} case 3.
\end{proof}

Next, we sketch the proof for the opposite of completeness -- the
soundness property of Algorithm~\ref{alg:1}.

\begin{theorem}
  Given \texttt{wait\_inbetween (M..N)} Algorithm~\ref{alg:1}
  \textbf{does not} give a value \texttt{d}, such that for \textbf{all}
  given reaction times $T$ in the range of $BCRT > T > WCRT$, $M \leq T
  \times d \leq N$ holds.
\end{theorem}

\begin{proof}
  The proof is the opposite of case 3 in Theorem~\ref{th:1}.
\end{proof}

The proofs for correctness, completeness, and soundness for
Algorithm~\ref{alg:2} are similar to those of Algorithm~\ref{alg:1}.

%%% Local Variables: 
%%% mode: latex
%%% TeX-master: "paper"
%%% End: 


\section{Experimental results}
\label{sec:experimental-results}




%%% Local Variables: 
%%% mode: latex
%%% TeX-master: "paper"
%%% End: 



\section{Discussion}
\label{sec:disc-perc-limit}

We dedicate this section to discuss the details of the wait construct
semantics, their usage and the advantages and limitations of our
proposed technique to introducing these real-time constructs in a
reactive setting.

\subsection{Programming using the wait\_exact construct -- the
  periodic reactions}
\label{sec:progr-using-delay}

% All three delay constructs are useful for programming real systems. % The
% first form of the delay construct $delay(M..N)$ is used to program
% non-determinism like in the HRCTS system and a real printer-spooler
% embedded system, etc. The second form is useful for programming low
% priority periodic tasks.

% \subsubsection{Non-deterministic time}
% \label{sec:non-determ-time}

% Many real-world systems require non-deterministic timing
% constructs. Where the exact real-time delay is not known or should not
% be known apriori. One such system is our motivating example -- the human
% response time system (HRTCS). Another one we presented was a real
% printer-spooler embedded controller in
% Section~\ref{sec:experimental-results}.

% \subsubsection{Timeout}
% \label{sec:timeout}

% There are many instances when one would like to wait on an input from
% environment for only a specified amount of time. This can be programmed
% as:

% \begin{figure}[h!]
%   \centering
% 			\vspace{-10pt}
% 			\begin{lstlisting}[style=sysj,basicstyle=\normalsize\ttfamily,morekeywords={await,emit,trap,pause,exit,delay}]
% // timeout after 1ms.
% trap(T){{await(A);}||
%         {delay(1 ms);exit(T);}}
% \end{lstlisting}
%   \caption{Programming low priority timeout tasks with delay statements}
%   \label{fig:timeout}
% 			\vspace{-10pt}
% \end{figure}

% In the above case, the program waits for signal \texttt{A} from the
% environment for \textit{at least} \texttt{1 ms}, but it may wait longer
% if other reactions/CDs are scheduled for execution. This second variant
% of delay statement is very useful for programming low priority tasks.

% \subsubsection{Periodic reactions}
% \label{sec:periodic-reactions}

% The third variant is useful for hard-real time guarantees, e.g., a
% reaction, or a whole CD, can be programmed to run periodically like so:

\begin{figure}[h!]
  \centering
	\vspace{-10pt}
        \begin{lstlisting}[style=sysj,basicstyle=\normalsize\ttfamily,morekeywords={emit,trap,pause,exit,wait_exact}]
          while(true){wait_exact (1 ms); emit S; //do something
          }
 \end{lstlisting}
  \caption{Programming periodic tasks with wait\_exact statement}
  \label{fig:periodic}
	\vspace{-10pt}
\end{figure}

The program in Figure~\ref{fig:preemp}, reproduced from
Section~\ref{sec:progr-using-exact}, is supposed to emit signal
\texttt{S} every \texttt{1 ms}. Such periodic reactions (or CDs) require
special consideration, like any other real-time periodic task. One
should be aware of the interaction of wait and other \texttt{pause}
constructs in the body of the reaction (or CD). For example, consider
the (perceived) periodic program and its rewrite below.

\begin{figure}[h!]
  \centering
  \begin{SubFloat}{\label{pp:a}Original program}
    \centering
			\begin{lstlisting}[style=sysj,basicstyle=\normalsize\ttfamily,morekeywords={emit,trap,pause,exit,wait_exact}]
while(true) {
 wait_exact(1 ms);
 //extra pause
 pause;
 emit S;
}
\end{lstlisting}
\end{SubFloat}
\hspace{9pt}
  \begin{SubFloat}{\label{pp:b}Rewritten program}
			\begin{lstlisting}[style=sysj,basicstyle=\normalsize\ttfamily,morekeywords={emit,trap,pause,exit,delay}]
while(true) {
 while(true) { 
  trap(T) {
   int x = 0;
   while(true) {
    x = x + 1;
    pause;
    if (x == d)
     exit(T);
   }
  }
  //extra pause
  pause;
  emit S;
 }
}
\end{lstlisting}
  \end{SubFloat}
\caption{Extra pause statements in periodic tasks}
  \label{fig:periodic2}
\end{figure}

The system designer needs to be aware that just introducing a wait
construct does not make the reaction (or CD) periodic. The program on
the left does \textit{not} emit signal \texttt{S} every \texttt{1 ms},
because of the extra \texttt{pause} statement. Introducing additional
pauses introduces more logical ticks, consequently increasing the
real-time wait period. Static analysis techniques from real-time
community can be used to automatically determine the wait specification,
in such cases, but describing these techniques is out of the scope of
this work.

\subsection{Need for time analyzable platforms}
\label{sec:need-time-analyzable}

The proposed solution relies on time analyzable platforms, i.e.,
platforms where WCRT and BCRT can be calculated using static analysis
tools. Many might consider this to be too much of a restriction. In this
section we debunk this \textit{perceived} disadvantage.

\subsubsection{Interaction of timers and reactive constructs}
\label{sec:inter-timers-react}

% As mentioned previously, the proposed solution might need
% over-approximation techniques (Algorithm~\ref{alg:2}) to guarantee
% timing requirements (cf. Section~\ref{sec:extend-tehcn-vari}). This
% might give the readers an impression that timer based solution proposed
% for standard real-time operating systems should be utilized for
% GALS/synchronous programs. This is a fallacy. Bourke et
% al~\cite{Bourke2009a} have investigated this technique
% unsuccessfully. We elaborate upon the interplay of timers and reactivity
% to make the problem of using timers explicit.

Preemption plays an important role in reactive languages. One needs to
carefully consider the interplay of wait constructs semantics with the
preemption semantics of reactive languages. Previous attempts at
incorporating postponement using external timers has only been partially
successful, because of the complex interplay between real-time and
preemption. Consider the simple example below, which models real-time
using external timers as in~\cite{rsh94}.

\begin{figure}[h!]
  \centering
	\vspace{-10pt}
		\begin{lstlisting}[style=sysj,basicstyle=\normalsize\ttfamily,morekeywords={emit,trap,pause,exit,delay,suspend}]
suspend(S) {
 emit START_TIMER(10); 
 await (TIMER);emit O1;
}
		\end{lstlisting}
  \caption{Interaction of preemption and external timers}
  \label{fig:preemp}
	\vspace{-10pt}
\end{figure}

As identified in~\cite{Bourke2009a} \texttt{suspend} does not play well
with the external timer. The program in Figure~\ref{fig:preemp} sends a
signal to an external timer and waits for 10 ms to pass by. Consider
what happens when signals \texttt{S} and \texttt{TIMER} occur in the
same logical tick, the \texttt{await} statement is never executed (due
to suspend) and hence, we enter a deadlock. Such problems are completely
avoided in our technique, because we convert the real-time wait
constructs into logical waits (\texttt{pause} constructs), which
interact well with preemption.  Furthermore, synchronous/GALS languages
have always been targeted at real-time analyzable platforms and
constrained
environments~\cite{DBLP:journals/pieee/SifakisTY03,boldt07}. One can use
our technique within such a setting with ease.

\subsubsection{The timer resolution problem}
\label{sec:resolution-real-time}

\begin{figure}[h!]
  \centering
	\vspace{-10pt}
		\begin{lstlisting}[style=sysj,basicstyle=\normalsize\ttfamily,morekeywords={emit,trap,pause,exit,delay,suspend}]
{
 //reaction R1
 emit START_TIMER(1 ms); 
 await (TIMER);emit O1;
}
|| //synchronous parallel
{
 //reaction R2
 // do something
}
		\end{lstlisting}
  \caption{Resolution of external timers}
  \label{fig:resolution}
	\vspace{-10pt}
\end{figure}

There is yet another problem with timer based systems. Let us consider
the program in Figure~\ref{fig:resolution}. On a cursory look this
example should work fine. An external timer is started in reaction R1,
which counts down from 1 ms. Once this time has elapsed, a signal
\texttt{TIMER} is generated from this external timer, which should emit
O1 in turn. There is no \texttt{suspend} construct encapsulating the
timer and hence all seems fine. Let us now consider what happens to the
whole CD, including reaction R2; while reaction R1 is waiting for the
\texttt{TIMER} signal, reaction R2 is making progress. Input signals can
only be captured from the environment at the logical tick boundaries
(cf. Section~\ref{sec:mapping-logical-time}). In
Figure~\ref{fig:resolution}, reaction R2 determines the length of the
tick, because it is performing the heavier computation. Consider what
happens if the external timer generates the \texttt{TIMER} signal, but
the system is still performing computation and the logical tick has not
yet finished -- we \textit{miss} this \texttt{TIMER} signal from the
environment. Thus, even in the absence of the \texttt{suspend}
construct, we are not guaranteed that the program will capture the
external timer signal. In fact, the WCRT needs to be smaller than 1 ms
for the above program to perform as expected. But, this again implies
need for a time analyzable platform even when using external timers.

In general we observe that the WCRT determines the lowest resolution of
any real-time construct whether it be external timer independent, like
the proposed technique, or external timer dependent, like above. This
discovery is one of the key insights for proposing the solutions
described in this paper.

% \subsection{Other semantic discussion}
% \label{sec:other-semant-dissc}

\subsection{Interaction of channel communication and wait constructs}
\label{sec:inter-chann-comm}

Channels, used for communication between reactions in asynchronously
running CDs, are an addition in the SystemJ language. Like interaction
of preemption and waits, conversion of real-time wait constructs to
logical ticks also interacts well with channel rendezvous, because the
semantics of interaction are well defined~\cite{amal10}. More
importantly, we need to consider the interplay of channel communication
and WCRT/BCRT analysis. Since, channel communication does not stop
logical time (see~\cite{amal10}) WCRT/BCRT are unaffected by channel
communication. % The response time to input signals \textit{is} though!
% But, we are unconcerned with the response time, analysis in this paper
% and it remains a future research avenue.

% \subsubsection{Interaction of synchronous parallelism and delay}
% \label{sec:inter-synchr-parall}

% {\color{red}Maybe we should talk about how the control flow in other
%   reactions and CDs is not affected by the delay statement.}



%%% Local Variables: 
%%% mode: latex
%%% TeX-master: "paper"
%%% End: 

% \vspace{-15pt}
\section{Related work}
\label{sec:related-work}



%%% Local Variables: 
%%% mode: latex
%%% TeX-master: "paper"
%%% End: 



\section{Conclusion}
\label{sec:concl-future-work}

In this paper we have described a novel way to introduce real-time in
\textit{Globally Asynchronous Locally Synchronous} (GALS) languages, in
particular SystemJ, and their subset the synchronous languages like
Esterel. The fundamental idea is to convert real-time delay into logical
delays (\texttt{pause} construct) that interact well with rest of the
constructs in these languages, especially, preemption and channel
communication. We do \textit{not} need or use external timers to
introduce these real-time delays, thereby resolving the problems of
interaction between external timers and preemption. Moreover, we also do
not require specific timer resolutions, thereby making our solution
elegant (in the spirit of GALS/synchronous programming) and flexible. We
rewrite the delay construct into reactive kernel constructs, calculate
the \textit{Worst} and \textit{Base} case reaction times and then
determine \textit{delay} \texttt{d}, number of logical ticks, which is
invariant to the \textit{elasticity} of logical time. To our knowledge
we are the first to introduce non-deterministic delays in such languages
and show their usefulness in designing real-world systems.


%%% Local Variables: 
%%% mode: latex
%%% TeX-master: "paper"
%%% End: 


\bibliographystyle{abbrv}
\bibliography{main}  % sigproc.bib is the name of the Bibliography in this case
\end{document}
